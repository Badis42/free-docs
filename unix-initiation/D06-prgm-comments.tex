%%%%%%%%%%%%%%%%%%%%%%%%%%%%%%%%%%%%%%%%%%%%%%%%%%%%%%%%%%%%%%%%%%%%%%%%
%                                                                      %
% This program is free software; you can redistribute it and/or modify %
% it under the terms of the GNU General Public License as published by %
% the Free Software Foundation; either version 2 of the License, or    %
% (at your option) any later version.                                  %
%                                                                      %
% This program is distributed in the hope that it will be useful,      %
% but WITHOUT ANY WARRANTY; without even the implied warranty of       %
% MERCHANTABILITY or FITNESS FOR A PARTICULAR PURPOSE.  See the        %
% GNU General Public License for more details.                         %
%                                                                      %
% You should have received a copy of the GNU General Public License    %
% along with this program; if not, write to the Free Software          %
% Foundation, Inc., 51 Franklin St, Fifth Floor, Boston,               %
% MA  02110-1301  USA                                                  %
%                                                                      %
%%%%%%%%%%%%%%%%%%%%%%%%%%%%%%%%%%%%%%%%%%%%%%%%%%%%%%%%%%%%%%%%%%%%%%%%
%
%	$Id$
%

\setcounter{remarque-cnt}{1}
\setcounter{example-cnt}{1}
\chapter{Les commentaires et les techniques d'ex{\'e}cution}
\thispagestyle{fancy}

%%%%%%%%%%%%%%%%%%%%%%%%%%%%%%%%%%%%%%%%%%%%%%%%%%%%%%%%%%%%%%%%
\section{\label{commexec-comments}Les commentaires}

\begin{definition}{Syntaxe}
\begin{verbatim}
# <ligne de commentaire>
commande	# commentaire
\end{verbatim}
\end{definition}

Le signe \index{#@\texttt{\#}}<<~\verb=#=~>> marque le d{\'e}but d{'}une zone de \index{commentaire}commentaire du script.
Elle se termine {\`a} la fin de la ligne. Tout ce qui suit ce caract{\`e}re est donc ignor{\'e} par le shell.

%%%%%%%%%%%%
\section{Interpr{\'e}tation sp{\'e}ciale du signe <<~\texttt{\#}~>> sur la premi{\`e}re ligne d'un shell script}

Si l'on satisfait les conditions suivantes~:
\begin{itemize}
	\item	le signe <<~\verb=#=~>> se trouve en premi{\`e}re ligne du fichier shell script {\bf et}
			sur la premi{\`e}re colonne,
	\item	il est suivi du caract{\`e}re <<~\verb=!=~>>,
	\item	la s{\'e}quence <<~\verb=#!=~>> est suivie du nom de l'ex{\'e}cutable d'un shell (chemin absolu),
\end{itemize}

alors, cette ligne permettra d'avoir la description du shell {\`a} lancer au moment de l'ex{\'e}cution.

Le processus mis en place est le suivant lorsqu'une telle ligne est trouv{\'e}e~:
\begin{itemize}
	\item	lancement du shell sp{\'e}cifi{\'e} sur la premi{\`e}re ligne,
	\item	lecture du fichier et {\'e}valuation/ex{\'e}cution des commandes par le shell que l'on
			vient le lancer.
\end{itemize}

Cette d{\'e}marche est obligatoire pour tout shell script (cf. r{\`e}gle 10 de la section \ref{pgm-intro-basic-rules}).

Cette technique est donc utilis{\'e} pour l'ensemble des langages interpr{\'e}t{\'e}s sous {\Unix} comme~:
\begin{itemize}
	\item	les diff{\'e}rentes variantes du shell (Bourne Shell, C-Shell, Korn Shell, etc.),
	\item	des utilitaires admettant des fichiers de requ{\^e}tes comme <<~\texttt{egrep}~>> (cf. section \ref{adv-fltrs-egrep}),
			<<~\texttt{sed}~>> (cf. section \ref{adv-fltrs-sed}) et <<~\texttt{awk}~>> (cf. section \ref{adv-fltrs-awk}),
	\item	des langages interpr{\'e}t{\'e}s comme <<~\texttt{perl}~>>\cite{learning-perl,programming-perl,advpgm-perl}.
\end{itemize}

Nous pourrons donc avoir les cas de figure suivants~:

\begin{longtable}{|l|p{10cm}|}
	\hline
	\multicolumn{2}{|r|}{Suite de la page pr{\'e}c{\'e}dente $\cdots$} \\
	\hline
	\multicolumn{1}{|c|}{Syntaxe}	&
	\multicolumn{1}{|c|}{Signification}	\\
	\hline
\endhead
	\hline
	\multicolumn{1}{|c|}{Syntaxe}	&
	\multicolumn{1}{|c|}{Signification}	\\
	\hline
\endfirsthead
	\hline
	\multicolumn{2}{|r|}{Suite page suivante $\cdots$} \\
	\hline
\endfoot
	\hline
\endlastfoot
	\hline
		\verb=#!/bin/sh=	&
			Tout ce qui va suivre ob{\'e}it {\`a} la syntaxe Bourne Shell. Le processus
			{\`a} lancer pour ex{\'e}cuter et {\'e}valuer les instructions suivantes ex{\'e}cutera le
			programme <<~\verb=/bin/sh=~>>.
			\\
	\hline
		\verb=#!/bin/csh=	&
			Tout ce qui va suivre ob{\'e}it {\`a} la syntaxe C Shell. Le processus
			{\`a} lancer pour ex{\'e}cuter et {\'e}valuer les instructions suivantes ex{\'e}cutera le
			programme <<~\verb=/bin/csh=~>>.
			\\
	\hline
		\verb=#!/bin/ksh=	&
			Tout ce qui va suivre ob{\'e}it {\`a} la syntaxe Korn Shell. Le processus
			{\`a} lancer pour ex{\'e}cuter et {\'e}valuer les instructions suivantes ex{\'e}cutera le
			programme <<~\verb=/bin/ksh=~>>.
			\\
	\hline
		\verb=#!/bin/egrep=	&
			Tout ce qui va suivre ob{\'e}it {\`a} la syntaxe de la commande <<~/bin/egrep~>>. Le processus
			{\`a} lancer pour ex{\'e}cuter et {\'e}valuer les instructions suivantes ex{\'e}cutera le
			programme <<~\verb=/bin/egrep=~>>.
			\\
	\hline
		\verb=#!/bin/sed=	&
			Tout ce qui va suivre ob{\'e}it {\`a} la syntaxe de la commande <<~\texttt{/bin/sed}~>>. Le processus
			{\`a} lancer pour ex{\'e}cuter et {\'e}valuer les instructions suivantes ex{\'e}cutera le
			programme <<~\verb=/bin/sh=~>>.
			\\
	\hline
		\verb=#!/bin/awk=	&
			Tout ce qui va suivre ob{\'e}it {\`a} la syntaxe de l'utilitaire <<~\texttt{awk}~>>. Le processus
			{\`a} lancer pour ex{\'e}cuter et {\'e}valuer les instructions suivantes ex{\'e}cutera le
			programme <<~\verb=/bin/awk=~>>
			\\
	\hline
		\verb=#!/usr/bin/perl=	&
			Tout ce qui va suivre ob{\'e}it {\`a} la syntaxe de Perl. Le processus
			{\`a} lancer pour ex{\'e}cuter et {\'e}valuer les instructions suivantes ex{\'e}cutera le
			programme <<~\verb=/usr/bin/perl=~>>
			\\
	\hline
		\multicolumn{2}{|l|}{etc.}
		\\
\end{longtable}

%%%%%%%%%%%%
\subsection{Les appels d'un shell script au niveau de la ligne de commandes}

La notion expos{\'e}e ici s'applique {\`a} tout type de programmes
\index{shell!script!appel, ex{\'e}cution}interpr{\'e}t{\'e}s sous {\Unix}. Nous ne traiterons ici que l'exemple
du Bourne Shell (<<~\texttt{/bin/sh}~>>).

\begin{definition}{Syntaxes}
\begin{longtable}{|l|p{5cm}|}
	\hline
	\multicolumn{2}{|r|}{Suite page suivante $\cdots$}	\\
	\hline
	\multicolumn{1}{|p{5cm}|}{Appel au niveau de la ligne de commande}	&
	\multicolumn{1}{|c|}{Descriptions -- Restrictions}	\\
	\hline
\endhead
	\hline
	\multicolumn{1}{|p{5cm}|}{Appel au niveau de la ligne de commande}	&
	\multicolumn{1}{|c|}{Descriptions -- Restrictions}	\\
	\hline
\endfirsthead
	\hline
	\multicolumn{2}{|r|}{Suite de la page pr{\'e}c{\'e}dente.}	\\
	\hline
\endfoot
	\hline
\endlastfoot
	\hline
	\texttt{shell\_script [arguments-programme]}	&
		<<~\texttt{shell\_script}~>> repr{\'e}sente le nom du fichier. Celui-ci doit {\^e}tre accessible en
		lecture {\bf et en ex{\'e}cution}.
		\\
	\hline
	\texttt{/bin/sh shell\_script [arguments-programme]}	&
		<<~\texttt{shell\_script}~>> repr{\'e}sente le nom du fichier. Celui-ci doit {\^e}tre accessible en
		lecture {\bf uniquement}.
		\\
	\hline
	\texttt{/bin/sh -vx shell\_script [arguments-programme]}	&
		<<~\texttt{shell\_script}~>> repr{\'e}sente le nom du fichier. Celui-ci doit {\^e}tre accessible en
		lecture. Chaque ligne du fichier est interpr{\'e}t{\'e}e et affich{\'e}e avant son ex{\'e}cution sur la
		sortie d'erreurs standard.
		\\
\end{longtable}
\end{definition}

\begin{remarque}
L'option <<~\texttt{-vx}~>> est propre au Bourne Shell, C Shell et Korn
Shell. Pour d'autres outils comme
\index{sed@\texttt{sed}}<<~\texttt{sed}~>>,
\index{awk@\texttt{awk}}<<~\texttt{awk}~>> ou <<~\texttt{perl}~>>,
reportez-vous {\`a} la page du manuel associ{\'e}e.
\end{remarque}
