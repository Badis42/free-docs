%%%%%%%%%%%%%%%%%%%%%%%%%%%%%%%%%%%%%%%%%%%%%%%%%%%%%%%%%%%%%%%%%%%%%%%%
%                                                                      %
% This program is free software; you can redistribute it and/or modify %
% it under the terms of the GNU General Public License as published by %
% the Free Software Foundation; either version 2 of the License, or    %
% (at your option) any later version.                                  %
%                                                                      %
% This program is distributed in the hope that it will be useful,      %
% but WITHOUT ANY WARRANTY; without even the implied warranty of       %
% MERCHANTABILITY or FITNESS FOR A PARTICULAR PURPOSE.  See the        %
% GNU General Public License for more details.                         %
%                                                                      %
% You should have received a copy of the GNU General Public License    %
% along with this program; if not, write to the Free Software          %
% Foundation, Inc., 51 Franklin St, Fifth Floor, Boston,               %
% MA  02110-1301  USA                                                  %
%                                                                      %
%%%%%%%%%%%%%%%%%%%%%%%%%%%%%%%%%%%%%%%%%%%%%%%%%%%%%%%%%%%%%%%%%%%%%%%%
%
%	$Id$
%

\section{Commandes de base de l'{\'e}diteur {\tt emacs}}
\renewcommand{\arraystretch}{1.5}

%%%%%%%%%%%%%%%%%%%%%%%%%%
\subsection{Introduction}

\index{emacs@\texttt{emacs}}"{\tt emacs}" est un {\'e}diteur {\sl pleine page} utilisable notamment sur
{\Unix}, {\OpenVMS}, {\DOS}, {\Windows}, {\MacOS}, etc.

Il fournit des fonctionnalit{\'e}s puissantes~:
\begin{itemize}
	\item	par des combinaisons de touches (notamment par emploi des touches
			"\ctrlkey" et "\esckey",
	\item	des menus d{\'e}roulants accessibles via la souris, sur des terminaux
			graphiques ou par une s{\'e}quence de touches, sur les terminaux
			{\ASCII}.
\end{itemize}

Par exemple, "{\tt emacs}" permet la recherche ou le remplacement
d'une cha{\^\i}ne de caract{\`e}res dans un programme ou la
visualisation de plusieurs fichiers en m{\^e}me temps par utilisation
de cl{\'e}s de fonctions pr{\'e}d{\'e}finies.

Le tableau \ref{emacs-cmds-base} donne une liste des commandes de base
de l'{\'e}diteur.

\begin{table}[hbtp]
\begin{tabular}{|p{6cm}|l|}
	\hline
		\multicolumn{1}{|c|}{Fonctionnalit{\'e}}		&
		\multicolumn{1}{|c|}{Commande / Touches}	\\
	\hline \hline
		Appel de "{\tt emacs}"			&
			{\tt emacs {\sl fichier}}		\\
	\hline
		Quitter "{\tt emacs}"			&
			\control{x} \control{c}			\\
	\hline
		Sauvegarder le fichier courant sous le m{\^e}me nom		&
			\control{x} \control{s}\footnote{cf. remarque \ref{emacs-control-s}} 		\\
											&
			ou \control{x} \control{w} 		\\
											&
			puis \returnkey					\\
	\hline
	Sauvegarder le fichier courant sous un nom diff{\'e}rent	&
		\control{x} \control{w} {\sl nouveau\_nom}			\\
															&
		puis \returnkey										\\
	\hline
	Documentation en ligne		&
		\control{h}				\\
	\hline
\end{tabular}
\caption{\label{emacs-cmds-base}Commandes de base de "{\tt emacs}"}
\end{table}

\begin{remarque}
Si un fichier sauvegard{\'e} {\'e}crase un fichier pr{\'e}{\'e}xistant,
ce dernier sera automatiquement sauvegard{\'e} sous un fichier
de m{\^e}me nom suivi du caract{\`e}re "\verb=~=".
\end{remarque}

\begin{remarque}
Quand un enregistrement du fichier en entr{\'e}e est plus long
que la ligne d'{\'e}cran, la ligne courante se termine par
le caract{\`e}re "\verb=\=" et l'enregistrement continue sur la ligne
suivante.
\end{remarque}

%%%%%%%%%%%%%%%%%%%%%%%%%%
\subsection{Organisation de l'{\'e}cran}

L'organisation de l'{\'e}cran d'"{\tt emacs}" se d{\'e}compose de la fa\c{c}on suivante~:
\begin{description}
	\item[{\rm Position du curseur}]\mbox{}\\
		C'est {\`a} partir de ce point que seront execut{\'e}es les commandes "{\tt emacs}".
	\item[{\sl Echo area}]\mbox{}\\
		Zone en bas de l'{\'e}cran o{\`u} sont affich{\'e}s les messages relatifs {\`a} des commandes
		(sauvegarde, messages d'erreur, intervention utilisateur). Cette zone
		correspond au {\sl buffer} "{\tt LSE\$MESSAGE}" de l'{\'e}diteur "{\tt LSEDIT}"
		d'{\OpenVMS}.
	\item[{\rm Affichage du {\sl buffer}}]\mbox{}\\
		Cette ou ces zones permettent de visualiser le contenu du ou des fichiers
		{\`a} {\'e}diter. Elles correspondent aux {\sl buffers} associ{\'e}s {\`a} des fichiers
		de l'{\'e}diteur "{\tt LSEDIT}" d'{\OpenVMS}.
\end{description}

La barre associ{\'e}e {\`a} un {\sl buffer} contient les informations suivantes~:
\begin{itemize}
	\item	Elle indique si le {\sl buffer} courrant a {\'e}t{\'e} modifi{\'e} ou non.
			Si le buffer n'a pas {\'e}t{\'e} modifi{\'e}, elle commencera par~:\\[2ex]
			\begin{center}
			\fbox{{\tt -----Emacs:} {\sl fichier}}
			\end{center}
			Dans le cas contraire, elle commencera par~:\\[2ex]
			\begin{center}
			\fbox{{\tt --**-Emacs:} {\sl fichier}}
			\end{center}
	\item	Chaque {\sl buffer} peut {\^e}tre associ{\'e} {\`a} un langage de programmation.
			Cette op{\'e}ration est effectu{\'e}e automatiquement lors de
			l'ouverture du fichier apr{\`e}s son analyse par
			"{\tt emacs}". Si tel est le cas, le langage associ{\'e}
			sera affich{\'e} sous la forme suivante~:\\[2ex]
			\begin{center}
			\fbox{{\tt (language)}}
			\end{center}
	\item	Enfin, "{\tt emacs}" indiquera la position courrante
			du curseur dans le fichier (num{\'e}ro de ligne et
			de colonne) et sa position relative (en pourcentage).
\end{itemize}
Pour plus de pr{\'e}cisions sur la notion de "{\sl buffers}", reportez-vous
{\`a} la section \ref{emacs-buffers}.
			
%%%%%%%%%%%%%%%%%%%%%%%%%%
\subsection{Cl{\'e}s de Fonction}

Une cl{\'e} de fonction peut {\^e}tre~:
\begin{itemize}
	\item	une combinaison simple de la forme \control{{\sl x}};
			par exemple \control{r} pour une recherche arri{\`e}re.

	\item	une combinaison multiple de la forme \control{{\sl x}} \control{{\sl y}};
			par exemple, pour quitter "{\tt emacs}", il faudra faire la s{\'e}quence
			de touches suivante "\control{x}" puis "\control{c}". Par convention,
			les cl{\'e}s de fonction multiples commencent par l'une des
			s{\'e}quences suivantes~:
			\begin{itemize}
				\item	\control{x}
				\item	\control{c}
				\item	\control{h}
			\end{itemize}
\end{itemize}

Tout d{\'e}but de commande multiple (ou passage d'argument) peut {\^e}tre annul{\'e} en cours
de saisie par la commande \control{g}. Ceci est aussi valable lors du passage
d'arguments par l'utilisateur.

\begin{remarque}
\label{emacs-control-s}
Il existe un certain nombre de cl{\'e}s de fonctions pr{\'e}d{\'e}finies,
mais l'utilisateur peut d{\'e}finir ses propres cl{\'e}s
de fonctions, soit parce que la fonctionnalit{\'e} n'existe
pas sous "{\tt emacs}", soit parce que la s{\'e}quence de touche {\`a}
laquelle elle est normalement attribu{\'e}e est inaccessible
sur certaines types de terminaux passifs, par exemple \control{s} pour
une recherche, est inaccessible sur les consoles VT).
\end{remarque}

"{\tt emacs}" est dot{\'e} d'un certain nombre de fonctions pr{\'e}d{\'e}finies,
qui sont en fait des associations entre des touches de fonctions
et des fonctions standards reconnus par l'{\'e}diteur de texte, par
exemple "{\tt scroll-down}", "{\tt kill-word}", "{\tt search-forward}", etc.
Il existe un fichier d'initialisation, o{\`u} "{\tt emacs}" va
lire les d{\'e}finitions utilisateurs avant toute entr{\'e}e
dans l'{\'e}diteur. Ce fichier s'appelle "{\tt .emacs}" et est
situ{\'e} dans votre r{\'e}pertoire de connexion. Il est bas{\'e} sur le langage "{\it lisp}",
langage de programmation utilis{\'e} pour l'ensemble des fichiers de configuration d'"{\tt emacs}".
Ce langage {\'e}tait, {\`a} l'origine, associ{\'e} {\`a} l'intelligence artificielle et a {\'e}t{\'e} le pr{\'e}curseur
de la programmation objet.

\begin{example}
{\sl Exemple de fichier "{\tt .emacs}"}
\begin{quote}
\begin{verbatim}
(define-key global-map "\C-xf" 'isearch-forward)
(define-key global-map "\C-xl" 'goto-line)
\end{verbatim}
\end{quote}
La premi{\`e}re ligne associe la s{\'e}quence \control{x} \key{f} {\`a} "{\it recherche avant}".
La seconde ligne associe la s{\'e}quence \control{x} \key{l} {\`a} "{\it positionnement {\`a} une ligne}".
\end{example}

\begin{definition}{{\bf Attention}}
Si la cl{\'e} de fonction est d{\'e}j{\`a} affect{\'e} par le syst{\`e}me
{\`a} une autre fonction, l'affectation pr{\'e}c{\'e}dente
est {\'e}cras{\'e}e par l'affectation utilisateur.
\end{definition}

%%%%%%%%%%%%%%%%%%%%%%%%%%
\subsection{\label{emacs-buffers}Notion de buffers}


%%%%%%%%%%%%
\subsubsection{Introduction}

Chaque fichier {\'e}dit{\'e} est stock{\'e} dans un buffer,
et au cours d'une session peut {\^e}tre rappel{\'e} {\`a}
tout moment, par menu d{\'e}roulant ou par la commande \control{x}
\key{b} {\sl nom\_de\_buffer}. Un buffer a en g{\'e}n{\'e}ral le m{\^e}me
nom que le fichier qu'il repr{\'e}sente.

La commande permettant de conna{\^\i}tre la liste des buffers
est \control{x} \control{b}. Une commande peut {\^e}tre pass{\'e}e
en t{\^e}te de chaque ligne pour s{\'e}lectionner un buffer
(mode pleine-page, mode fen{\^e}tre, etc.). Une ast{\'e}risque
en colonne 1 indique que le buffer correspondant a {\'e}t{\'e}
modifi{\'e}, un "\verb=%=" indique un acc{\`e}s en mode lecture uniquement.
Pour avoir la liste des fonctionnalit{\'e}s disponibles, taper
\key{?} en t{\^e}te de ligne.

La commande permettant de supprimer un buffer est \control{x} \key{k}. Pour
supprimer plusieurs buffers, activer la liste des buffers (commande
\control{X} \control{B}), taper \key{d} au d{\'e}but des lignes indiquant les
buffers {\`a} supprimer. La commande sera effective lorsque
vous taperez \key{x}.

Si vous cr{\'e}ez un nouveau buffer (\control{x} \key{b}), il n'y aura aucun lien
entre le buffer ouvert et un {\'e}ventuel fichier disque portant le m{\^e}me nom. Encore
une fois, un buffer ne repr{\'e}sente qu'un espace de travail.

\begin{remarque}
Supprimer un buffer
n'entra{\^\i}ne pas suppression du fichier disque. Seule la copie
du fichier dans l'espace de travail "{\tt emacs}" est supprim{\'e}e.
\end{remarque}

%%%%%%%%%%%%
\subsubsection{Les "{\sl mini-buffers}"}

Le "{\sl mini-buffer}" est une facilit{\'e} "{\tt emacs}" pour lire les arguments
d'une commande (nom de fichier, commande interactive n{\'e}cessitant
une r{\'e}ponse de l'utilisateur, etc.). Il est par exemple utilis{\'e}
lorsque l'utilisateur d{\'e}sire ouvrir un nouveau "{\sl buffer}" associ{\'e} {\`a}
un fichier d{\'e}j{\`a} existant (commande \control{x} \key{f}). Un nom de fichier
devrait suivre cette commande. Or, si l'utilisateur saisit simplement
\control{x} \key{f}, "{\tt emacs}" r{\'e}clame le nom de fichier dans
la ligne d'{\'e}tat situ{\'e} en bas de l'{\'e}cran.

En cas d'h{\'e}sitation sur l'argument {\`a} entrer dans
le "{\sl mini-buffer}", une assistance permet d'aiguiller l'utilisateur~:
\begin{itemize}
	\item	"\verb=?=" affiche l'utilisation des noms possibles compte tenu de
			ce qui a d{\'e}j{\`a} {\'e}t{\'e} saisi. La fen{\^e}tre affich{\'e}e est
			accessible et les commandes "{\tt emacs}" sont actives ({\it scrolling} par
			exemple) ainsi que les commandes de s{\'e}lection de "{\sl buffer}"
			vues pr{\'e}c{\'e}demment.
	\item	\tabkey compl{\`e}te le texte dans la mesure du possible.
	\item	\spacekey compl{\`e}te le texte dans la mesure du possible mais limit{\'e} {\`a} un seul mot.
\end{itemize}

%%%%%%%%%%%%%%%%%%%%%%%%%%
\subsection{Utilisation de l'aide}

L'aide est activ{\'e} par  \control{H} suivi d'une  lettre ou d'un signe d{\'e}terminant
la nature de l'aide.

Le tableau \ref{emacs-help} liste les principales commandes acc{\'e}dant
aux rubriques de l'aide.

\begin{table}
\begin{tabular}{|l|p{10cm}|}
	\hline
		\multicolumn{1}{|c|}{S{\'e}quence de touche}	&
		\multicolumn{1}{|c|}{Rubrique / Fonction}	\\
	\hline \hline
		\control{h} \key{?}		&
		Liste des rubriques d'aide.			\\
	\hline \hline
		\control{H} {\sl cha{\^\i}ne} \key{A}	&
		Liste des commandes contenant le mot "{\sl cha{\^\i}ne}".	\\
	\hline
		\control{h} \key{b}	&
		Liste des raccourcis clavier disponibles.	\\
	\hline
		\control{h} \key{f} {\sl fonction}		&
		Description d'une fonction "{\tt emacs}".	\\
	\hline
		\control{h} \key{n}	&
		Nouveaut{\'e}s de la version actuelle d'"{\tt emacs}".	\\
	\hline
		\control{h} \key{t}	&
		Tutoriel\\
	\hline
\end{tabular}
\caption{\label{emacs-help}Principales commandes acc{\'e}dant aux rubriques d'aide d'"{\tt emacs}".}
\end{table}

%%%%%%%%%%%%%%%%%%%%%%%%%%
\subsection{Utilisation de quelques commandes}

%%%%%%%%%%%%%%
\subsubsection{D{\'e}placement dans le "buffer"~}

\begin{longtable}{|l|p{10cm}|}
	\hline
		\multicolumn{2}{|r|}{Suite de la page pr{\'e}c{\'e}dente $\cdots$}	\\
	\hline
		\multicolumn{1}{|c|}{S{\'e}quence de touche}	&
		\multicolumn{1}{|c|}{Description}	\\
	\hline \hline
\endhead
	\hline
		\multicolumn{1}{|c|}{S{\'e}quence de touche}	&
		\multicolumn{1}{|c|}{Description}	\\
	\hline \hline
\endfirsthead
	\hline
		\multicolumn{2}{|r|}{Suite page suivante $\cdots$}	\\
	\hline
\endfoot
	\hline
\endlastfoot
		\control{a}		&	D{\'e}but de ligne						\\
		\control{e}		&	Fin de ligne						\\
		\control{f}		&	D{\'e}placement un caract{\`e}re {\`a} droite	\\
		\control{b}		&	D{\'e}placement un caract{\`e}re {\`a} gauche	\\
		\escape{f}		&	D{\'e}placement un mot {\`a} droite			\\
		\escape{b}		&	D{\'e}placement un mot {\`a} gauche			\\
		\control{n}		&	D{\'e}placement une ligne en dessous	\\
		\control{p}		&	D{\'e}placement une ligne au dessus		\\
		\escape{$<$}	&	D{\'e}placement d{\'e}but de fichier		\\
		\escape{$>$}	&	D{\'e}placement fin de fichier			\\
		\control{v}		&	D{\'e}placement d'une page vers le bas	\\
		\escape{v}		&	D{\'e}placement d'une page vers le haut	\\
		\escape{a}		&	D{\'e}placement d{\'e}but phrase courante	\\
		\escape{e}		&	D{\'e}placement fin phrase courante		\\
		\escape{k}		&	Destruction jusqu'{\`a} la fin de la phrase courante.	\\
\end{longtable}

%%%%%%%%%%%%%%
\subsubsection{Recherche / Remplacement de caract{\`e}res}

\begin{longtable}{|l|p{10cm}|}
	\hline
		\multicolumn{2}{|r|}{Suite de la page pr{\'e}c{\'e}dente $\cdots$}	\\
	\hline
		\multicolumn{1}{|c|}{S{\'e}quence de touche}	&
		\multicolumn{1}{|c|}{Description}	\\
	\hline \hline
\endhead
	\hline
		\multicolumn{1}{|c|}{S{\'e}quence de touche}	&
		\multicolumn{1}{|c|}{Description}	\\
	\hline \hline
\endfirsthead
	\hline
		\multicolumn{2}{|r|}{Suite page suivante $\cdots$}	\\
	\hline
\endfoot
	\hline
\endlastfoot
		\control{s}	&	Recherche vers le bas	\\
		\control{r}	&	Recherche vers le haut	\\
		\escape{\%}	&	Remplacement par une cha{\^\i}ne (vers le bas).	\\
\end{longtable}

%%%%%%%%%%%%%%
\subsubsection{R{\'e}ponses possibles pour la recherche et le remplacement de caract{\`e}res}

\begin{longtable}{|l|p{10cm}|}
	\hline
		\multicolumn{2}{|r|}{Suite de la page pr{\'e}c{\'e}dente $\cdots$}	\\
	\hline
		\multicolumn{1}{|c|}{S{\'e}quence de touche}	&
		\multicolumn{1}{|c|}{Description}	\\
	\hline \hline
\endhead
	\hline
		\multicolumn{1}{|c|}{S{\'e}quence de touche}	&
		\multicolumn{1}{|c|}{Description}	\\
	\hline \hline
\endfirsthead
	\hline
		\multicolumn{2}{|r|}{Suite page suivante $\cdots$}	\\
	\hline
\endfoot
	\hline
\endlastfoot
	\spacekey	&	Remplacement de l'occurrence courante, positionnement {\`a} l'occurence
 					suivante.	\\
	\key{{\sc del}}	&	Positionnement {\`a} l'occurence suivante, sans remplacement.	\\
	\key{.} 		&	Remplacement occurrence courante et arr{\^e}t du processus.		\\
	\key{!}			&	Remplacement occurences restant jusqu'{\`a} la fin du texte.	\\
	\key{?}			&	Affichage des r{\'e}ponses possibles.							\\
\end{longtable}

%%%%%%%%%%%%%%
\subsubsection{Couper / Copier / Coller}
\begin{description}
	\item[{\rm Marquage de la r{\'e}gion}]\mbox{}\\
		Quelques commandes de "{\tt emacs}" g{\`e}rent sur une portion ({\`a}
		d{\'e}finir) du buffer. Par exemple, la copie ou suppression
		d'un bloc de texte n{\'e}cessite pr{\'e}alablement de d{\'e}finir
		le texte {\`a} couper. La r{\'e}gion sera l'espace entre
		le point du d{\'e}but de marquage et la position actuelle du
		curseur.\\[3ex]
	\item[{\rm D{\'e}finition du bloc de texte sur lequel va s'op{\'e}rer la commande}]\mbox{}\\
		\begin{tabular}{|l|p{6cm}|}
			\hline
			\multicolumn{1}{|c|}{S{\'e}quence de touche}	&
			\multicolumn{1}{|c|}{Description}	\\
			\hline \hline
				\control{{\sc space}}	&
				Marquage de d{\'e}but de bloc	\\
				 ou \control{0}			&	\\
			\hline
				\control{x} \control{x}	&
				Passage position curseur / D{\'e}but de marquage	\\
			\hline
				\control{x} \key{h}	&
				Activer le buffer entier comme r{\'e}gion.			\\
			\hline
				\escape{h}	&
				Activer le paragraphe courant comme r{\'e}gion.		\\
			\hline
		\end{tabular}\\[3ex]
	\item[{\rm Couper / Coller sur une r{\'e}gion}]\mbox{}\\
		\begin{tabular}{|l|p{6cm}|}
			\hline
			\multicolumn{1}{|c|}{S{\'e}quence de touche}	&
			\multicolumn{1}{|c|}{Description}	\\
			\hline \hline
				\control{w}	&	Destruction de la r{\'e}gion.	\\
				\control{y}	&	Insertion du texte coup{\'e}. Le curseur est positionn{\'e} apr{\`e}s le
								texte ins{\'e}r{\'e}.				\\
				\control{u} \control{y}	&
								Insertion texte coup{\'e}. Le curseur est positionn{\'e}; avant
								le texte ins{\'e}r{\'e}.			\\
			\hline
		\end{tabular}\\[3ex]
	\item[{\rm Couper / Coller sur d'autres {\'e}l{\'e}ments}]\mbox{}\\
		\begin{tabular}{|l|p{6cm}|}
			\hline
			\multicolumn{1}{|c|}{S{\'e}quence de touche}	&
			\multicolumn{1}{|c|}{Description}	\\
			\hline \hline
				\control{d}		&	Suppression du caract{\`e}re sous le curseur.	\\
				\key{{\sc del}}	&	Suppression caract{\`e}re avant le curseur.		\\
				\control{k}		&	Suppression jusqu'{\`a} la fin de la ligne.		\\
				\control{u} {\sl n} \control{k}	&
									Suppression de "{\sl n}" lignes .		\\
				\escape{d}		&	Suppression jusqu'{\`a} la fin du mot.			\\
				\escape{k}		&	Suppression jusqu'{\`a} la fin de la phrase.	\\
			\hline
		\end{tabular}\\[3ex]
	\item[{\rm Autres op{\'e}rations sur une r{\'e}gion}]\mbox{}\\
		\begin{tabular}{|l|p{6cm}|}
			\hline
			\multicolumn{1}{|c|}{S{\'e}quence de touche}	&
			\multicolumn{1}{|c|}{Description}	\\
			\hline \hline
				\control{x} \control{u}	&	Conversion de la r{\'e}gion en majuscules.	\\
				\control{x} \control{l}	&	Conversion de la r{\'e}gion en minuscules.	\\
				\control{x} \tabkey		&	Indentation d'une r{\'e}gion.				\\
			\hline
		\end{tabular}
\end{description}

\begin{definition}{{\bf ATTENTION}}
\begin{itemize}
	\item	Il existe un seul espace de stockage des textes coup{\'e}s
			pour l'ensemble des buffers. C'est ce qui permet, entre autres,
			de copier un bloc de texte d'un buffer et le dupliquer dans un
			autre buffer.\\[2ex]
	\item	Si plusieurs commandes de coupure de texte s'encha{\^\i}nent
			sans qu'une autre commande ne vienne interf{\'e}rer, les textes
			coup{\'e}s seront stock{\'e}s ensemble, et une commande
			\control{Y} ins{\`e}rera l'ensemble des textes coup{\'e}s successifs.
			Par contre, si une commande a {\'e}t{\'e} execut{\'e}e
			entre deux coupures, y compris un d{\'e}placement, la prochaine
			insertion de texte concernera le dernier texte coup{\'e}.\\[2ex]
	\item	Un texte coup{\'e} pr{\'e}c{\'e}demment peut {\^e}tre
			ins{\'e}r{\'e}, m{\^e}me si ce n'est pas le dernier texte
			coup{\'e}. La proc{\'e}dure {\`a} suivre est la suivante~:
			\begin{enumerate}
				\item	\control{Y}
				\item	\escape{y} jusqu'{\`a} ce que le texte d{\'e}sir{\'e}
						soit ins{\'e}r{\'e} {\`a} l'{\'e}cran. \\
						{\large ou} \\
						\control{u} {\sl n} \control{Y} (Insertion du texte coup{\'e} "{\sl n}" fois
						pr{\'e}c{\'e}demment).
			\end{enumerate}	
\end{itemize}
\end{definition}

%%%%%%%%%%%%%%
\subsubsection{Op{\'e}rations sur les fen{\^e}tres}

"{\tt emacs}" peut partager l'{\'e}cran en deux ou plusieurs fen{\^e}res qui
affichent plusieurs parties d'un m{\^e}me buffer ou de plusieurs buffers. Il n'y a
pas identification entre fen{\^e}tre et buffer. Une fen{\^e}tre peut {\^e}tre d{\'e}truite, et le
buffer correspondant {\^e}tre encore actif.

Si un m{\^e}me buffer apparait dans plusieurs fen{\^e}tres, les modifications effectu{\'e}es
dans une fen{\^e}tre sont r{\'e}percut{\'e}es dans les autres. Les commandes les plus fr{\'e}quentes
sont explicit{\'e}es dans le tableau suivant.

\begin{longtable}{|l|p{10cm}|}
	\hline
		\multicolumn{2}{|r|}{Suite de la page pr{\'e}c{\'e}dente $\cdots$}	\\
	\hline
		\multicolumn{1}{|c|}{S{\'e}quence de touche}	&
		\multicolumn{1}{|c|}{Description}	\\
	\hline \hline
\endhead
	\hline
		\multicolumn{1}{|c|}{S{\'e}quence de touche}	&
		\multicolumn{1}{|c|}{Description}	\\
	\hline \hline
\endfirsthead
	\hline
		\multicolumn{2}{|r|}{Suite page suivante $\cdots$}	\\
	\hline
\endfoot
	\hline
\endlastfoot
		\control{x} \key{2}	&
		Partage la fen{\^e}tre courante en 2 fen{\^e}tres horizontales.	\\
	\hline
		\control{x} \key{o}	&
		S{\'e}lection d'une autre fen{\^e}tre.	\\
	\hline
		\esckey \control{v}	&
		S{\'e}lection de la fen{\^e}tre suivante. \\
	\hline 
		\control{x} \key{4} \key{b} {\sl buffer}	&
		S{\'e}lection d'un buffer dans une autre fen{\^e}tre. \\
	\hline 
		\control{x} \key{4} \key{f} {\sl buffer}	&
		S{\'e}lection d'un fichier dans une autre fen{\^e}tre. \\
	\hline 
		\control{x} \key{4} \key{d} {\sl directory}	&
		S{\'e}lection du contenu d'un r{\'e}pertoire dans une autre fen{\^e}tre. \\
	\hline 
		\control{x} \key{\^{ }}	&
		Agrandissement de la fen{\^e}tre courante. \\
	\hline 
		\control{x} \key{0}	&
		Suppression de la fen{\^e}tre courante. \\
	\hline 
		\control{x} \key{1}	&
		Suppression toutes fen{\^e}tres sauf la fen{\^e}tre courante. \\
\end{longtable}


%%%%%%%%%%%%%%
\subsubsection{Autres commandes diverses}

\begin{longtable}{|l|p{10cm}|}
	\hline
		\multicolumn{2}{|r|}{Suite de la page pr{\'e}c{\'e}dente $\cdots$}	\\
	\hline
		\multicolumn{1}{|c|}{S{\'e}quence de touche}	&
		\multicolumn{1}{|c|}{Description}	\\
	\hline \hline
\endhead
	\hline
		\multicolumn{1}{|c|}{S{\'e}quence de touche}	&
		\multicolumn{1}{|c|}{Description}	\\
	\hline \hline
\endfirsthead
	\hline
		\multicolumn{2}{|r|}{Suite page suivante $\cdots$}	\\
	\hline
\endfoot
	\hline
\endlastfoot
	\control{l}	&	Rafraichissement {\'e}cran.	\\
	\hline
	\escape{!}	&	Execution d'une commande shell. Le r{\'e}sultat apparaitra dans une
					fen{\^e}tre "{\tt emacs}".	\\
\end{longtable}

%%%%%%%%%%%%%%%%%%%%%%%%%%
\subsection{Macros}

Une s{\'e}quence de traitements "{\tt emacs}" peut {\^e}tre m{\'e}moris{\'e}e
afin d'{\^e}tre rejou{\'e}e une ou plusieurs fois ult{\'e}rieurement.
Cette s{\'e}quence peut {\^e}tre nomm{\'e}e, sauvegard{\'e}e
et restaur{\'e}e au cours d'une autre session "{\tt emacs}".

\begin{quote}
{\tt \control{X} (}\\
{\sl D{\'e}finition de la macro}\\
{\sl liste des traitements emacs}\\
{\tt \control{X} )}
\end{quote}

\begin{tabular}{|l|p{7cm}|}
	\hline
		\multicolumn{1}{|c|}{S{\'e}quence de touche}	&
		\multicolumn{1}{|c|}{Description}	\\
	\hline \hline
		\control{x} \key{e}	&	Execution de la derni{\`e}re macro cr{\'e}{\'e}e.	\\
	\hline
		\multicolumn{1}{|p{5cm}|}{\control{u} {\sl n} \control{x} \key{e} \returnkey}		&
		Execution {\sl n} fois de la derni{\`e}re macro cr{\'e}{\'e}e.	\\
	\hline
		\escape{x} {\sl name-last-kbd-macro}	&
		Association d'un nom {\`a} la macro derni{\`e}rement d{\'e}finie.	\\
	\hline
\end{tabular}

Pour pouvoir r{\'e}utiliser cette macro au cours d'une autre
session, on peut~:
\begin{itemize}
	\item	Soit sauver sa d{\'e}finition dans un fichier externe qu'il
			faudra recharger avant utilisation (commande "\escape{x}~{\tt load-file}").
	\item	Soit sauver sa d{\'e}finition dans le fichier d'initialisation
			"{\tt .emacs}". Il suffit de l'ouvrir, de s'y positionner,
			et d'entrer la commande~:
			\begin{quote}
			\escape{x}~{\tt insert-kbd-macro}~\returnkey
			{\sl nom\_de\_macro}~\returnkey
			\end{quote}
			Lors d'une prochaine session "{\tt emacs}", la macro sera execut{\'e}e
			par la commande\escape{x} {\sl nom\_de\_macro}.
\end{itemize}

\begin{remarque}
Pour annuler une modification de texte, il existe la commande "{\tt undo}" ( repr{\'e}sent{\'e}e par
\control{x} \key{u}). Cette modification de texte est li{\'e}e {\`a} un
buffer. Elle peut {\^e}tre r{\'e}p{\'e}t{\'e}e plusieurs fois (manuellement ou par la sequence
\control{u} {\sl n} \control{x} \key{u}). Les modifications de texte sont m{\'e}morisables dans la limite
de 8000 caract{\`e}res par buffer.
\end{remarque}

\renewcommand{\arraystretch}{1}

