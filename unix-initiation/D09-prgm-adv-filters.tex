%%%%%%%%%%%%%%%%%%%%%%%%%%%%%%%%%%%%%%%%%%%%%%%%%%%%%%%%%%%%%%%%%%%%%%%%
%                                                                      %
% This program is free software; you can redistribute it and/or modify %
% it under the terms of the GNU General Public License as published by %
% the Free Software Foundation; either version 2 of the License, or    %
% (at your option) any later version.                                  %
%                                                                      %
% This program is distributed in the hope that it will be useful,      %
% but WITHOUT ANY WARRANTY; without even the implied warranty of       %
% MERCHANTABILITY or FITNESS FOR A PARTICULAR PURPOSE.  See the        %
% GNU General Public License for more details.                         %
%                                                                      %
% You should have received a copy of the GNU General Public License    %
% along with this program; if not, write to the Free Software          %
% Foundation, Inc., 51 Franklin St, Fifth Floor, Boston,               %
% MA  02110-1301  USA                                                  %
%                                                                      %
%%%%%%%%%%%%%%%%%%%%%%%%%%%%%%%%%%%%%%%%%%%%%%%%%%%%%%%%%%%%%%%%%%%%%%%%
%
%	$Id$
%

\setcounter{remarque-cnt}{1}
\setcounter{example-cnt}{1}
\chapter{\label{adv-filters}Utilisation avanc{\'e}es de certains filtres}
\thispagestyle{fancy}

%%%%%%%%%%%%%%%%%%%%%%%%%%%%%%%%%%%%%%%%%%%%%%%%%%%%%%%%%%%%
\section{Introduction}

Dans la section \ref{bcpts-filters}, nous avons d{\'e}j{\`a} examiner la
syntaxe de quelques filtres. Nous allons maintenant regarder comment
certains d'entre eux utilisent les expressions r{\'e}guli{\`e}res. Nous
examinerons les filtres suivants~:
\begin{itemize}
	\item	<<~{\tt grep}~>>,
	\item	<<~{\tt egrep}~>>,
	\item	<<~{\tt fgrep}~>>.
\end{itemize}

%%%%%%%%%%%%%%%%%%%%%%%%%%%%%%%%%%%%%%%%%%%%%%%%%%%%%%%%%%%%
\section{\texorpdfstring{Utilisation avanc{\'e}e de <<~{\tt grep}~>>, <<~{\tt egrep}~>> et <<~{\tt fgrep}~>>}{Utilisation avanc{\'e}e de <<~grep~>>, <<~egrep~>> et <<~fgrep~>>}}

%%%%%%%%%%%%
\subsection{Introduction - Rappels}

Le filtre \index{grep@\texttt{grep}}<<~{\tt grep}~>> est utilis{\'e} pour rechercher, dans un fichier,
les lignes qui contiennent une cha{\^\i}ne de caract{\`e}res pr{\'e}cise.

\begin{definition}{Syntaxe~:}
{\tt grep [{\sl options}] {\sl expression-r{\'e}guli{\`e}re} [{\sl liste-de-fichiers}]}
\end{definition}

<<~{\tt grep}~>> compare les lignes sur son entr{\'e}e standard ou dans le
(les) fichier(s) {\`a} une expression r{\'e}guli{\`e}re plac{\'e}e sur la ligne de
commande. Il envoie ensuite sur la sortie standard toutes les lignes en
entr{\'e}e correspondant {\`a} l'expression r{\'e}guli{\`e}re. Si l'expression r{\'e}guli{\`e}re
contient des caract{\`e}res pouvant {\^e}tre interpr{\'e}t{\'e}s par le Shell, elle doit
{\^e}tre plac{\'e}e dans ce cas entre simple quotes <<~\verb='=~>> ou doubles quotes <<~\verb="=~>>.

\begin{definition}{Principales options}
\begin{tabular}{l@{\hspace{3ex}}p{10cm}}
	{\tt -c}	&	donne uniquement le nombre d'occurrences trouv{\'e}es,\\[0.5ex]
	{\tt -i}	&	ignore les majuscules/minuscules,\\[0.5ex]
	{\tt -l}	&	donne uniquement le nom des fichiers en entr{\'e}e dans lesquels
					au moins une occurrence a {\'e}t{\'e} trouv{\'e}e,\\[0.5ex]
	{\tt -n}	&	affiche le num{\'e}ro de la ligne o{\`u} se trouve
					l'occurrence,\\[0.5ex]
	{\tt -v}	&	donne les lignes dans lesquelles aucune occurrence n'a pas
					{\'e}t{\'e} trouv{\'e}e.\\
\end{tabular}
\end{definition}

\begin{remarque}
Les options <<~{\tt -c}~>>, <<~{\tt -v}~>> et <<~{\tt -n}~>> ne peuvent pas se
combiner entre elles.
\end{remarque}

\begin{example}
\begin{tabular}{l@{\hspace{2ex}}p{6cm}}
	\verb=grep dulac /etc/passwd=	&
		Recherche la chaine <<~{\tt dulac}~>> dans le fichier
		<<~{\tt /etc/passwd}~>>.	\\[0.5ex]
	\verb=grep '\\' albert=			&
		Recherche la chaine <<~\verb=\\=~>> dans le fichier <<~{\tt albert}~>>
		se trouvant dans le r{\'e}pertoire courant.	\\[0.5ex]
	\verb=grep \' arthur=			&
		Recherche la chaine <<~{\tt '}~>> dans le fichier <<~{\tt albert}~>>
		se trouvant dans le r{\'e}pertoire courant.	\\[0.5ex]
	\verb=grep "''" the.king=		&
		Recherche la chaine <<~\verb=''=~>> dans le fichier <<~{\tt the.king}~>>
		se trouvant dans le r{\'e}pertoire courant.	\\
\end{tabular}
\end{example}

%%%%%%%%%%%%
\subsection{\texorpdfstring{Utilisation de <<~{\tt grep}~>>}{Utilisation de <<~grep~>>}}

\index{grep@\texttt{grep}!utilisation}<<~{\tt grep}~>> accepte un ensemble restreint d'expressions r{\'e}guli{\`e}res,
en particulier, il n'accepte pas les symboles suivants~:\\
\begin{center}
\begin{tabular}{c@{\hspace{5ex}}c}
	\verb=+=	&	\verb=?=	\\
	\verb=|=	&	\verb=( )=	\\
\end{tabular}
\end{center}

<<~{\tt grep}~>> renvoie au shell un statut sur le r{\'e}sultat de sa recherche.
Les diff{\'e}rentes valeurs possibles sont~:

\begin{center}
\begin{tabular}
	{|@{\hspace{0.5ex}}c@{\hspace{0.5ex}}|@{\hspace{0.5ex}}p{5cm}@{\hspace{0.5ex}}|}
	\hline
		Code	&	Description	\\
	\hline \hline
		0		&	une occurrence au moins a {\'e}t{\'e} trouv{\'e}e.	\\[0.5ex]
		1		&	aucune occurrence n'a {\'e}t{\'e} trouv{\'e}e.		\\[0.5ex]
		2		&	il y a une erreur de syntaxe ou bien un
					des fichiers en entr{\'e}e n'est pas
					accessible								\\
	\hline
\end{tabular}
\end{center}

\begin{remarque}
Un statut <<~0~>> est {\'e}quivalent {\`a} <<~{\sl Vrai}~>>.\\
Un statut non nul {\'e}quivaut {\`a} <<~{\sl Faux}~>>.
\end{remarque}

On peut donc utiliser ce code dans les scripts lors de tests. Par exemple,
il est possible d'afficher un message lorsque la recherche a {\'e}t{\'e} fructueuse
ou non. Pour cela, la sortie standard de la commande <<~{\tt grep}~>> sera
redirig{\'e}e dans un fichier sp{\'e}cial et seul le statut d'ex{\'e}cution sera
utilis{\'e}.

\begin{example}
\begin{alltt}
if grep expression fichier >/dev/null 2>&1; then
    echo "Gagn{\'e}"
else
    echo "Perdu"
fi
\end{alltt}
Dans cet exemple, la commande <<~{\tt grep}~>> effectue sa recherche
dans <<~{\tt fichier}~>>. Tout message d'erreur ou r{\'e}sultat d'affichage
n'appara{\^i}tra pas {\`a} l'{\'e}cran. Si la recherche donne un r{\'e}sultat
(<<~\verb,$? = 0,~>>), le message <<~{\tt Gagn{\'e}}~>> sera affich{\'e}. Si la
recherche est infructueuse ou s'il se produit une erreur, <<~{\tt Perdu}~>>
sera affich{\'e}.
\end{example}

\begin{example}
\begin{verbatim}
grep '^[0-9][0-9]' fichier
\end{verbatim}
Dans cet exemple, on fait appel {\`a} la notion d'expression r{\'e}guli{\`e}re
(cf. \ref{reg-exp}). L'expression <<~\verb=^[0-9][0-9]=~>> se d{\'e}compose
de la fa\c{c}on suivante~:\\[1ex]
\begin{tabular}{l@{\hspace{2ex}}p{10cm}}
	\verb=^=		&	D{\'e}but de ligne.	\\[0.5ex]
	\verb=[0-9]=	&	Caract{\`e}re quelconque compris entre les codes
						{\ASCII} de 0 et de 9. Ces caract{\`e}res sont~: 0, 1, 2,
						3, 4, 5, 6, 7, 8, 9. Par cons{\'e}quent <<~\verb=[0-9]=~>>
						repr{\'e}sente un chiffre quelconque. Celui-ci doit
						appara{\^\i}tre une seule fois juste apr{\`e}s le d{\'e}but de
						ligne	\\[0.5ex]
	\verb=[0-9]=	&	Le second <<~\verb=[0-9]=~>> repr{\'e}sente lui-aussi
						un chiffre quelconque. Celui-ci doit appara{\^\i}tre une
						seule fois juste apr{\`e}s le pr{\'e}c{\'e}dent chiffre.
\end{tabular}
Cette expression r{\'e}guli{\`e}re permet de d{\'e}crire toute chaine de caract{\`e}res
commencant par deux chiffres quelconque. La commande <<~{\tt grep}~>>
va donc extraire de <<~{\tt fichier}~>>, toutes les lignes commencant
par deux chiffres.
\end{example}

%%%%%%%%%%%%
\section{\texorpdfstring{\label{adv-fltrs-egrep}Utilisation de <<~{\tt egrep}~>>}{Utilisation de <<~egrep~>>}}

\index{egrep@\texttt{egrep}}\index{grep@\texttt{grep}!egrep@\texttt{egrep}}La
commande <<~{\tt egrep}~>> fonctionne comme la commande <<~{\tt grep}~>>.
Par contre~:
\begin{itemize}
	\item	elle accepte la d{\'e}finition compl{\`e}te des expressions r{\'e}guli{\`e}res
			y compris <<~\verb=+=~>>, <<~\verb=?=~>>, <<~\verb=|=~>>
			et <<~\verb=( )=~>>.
	\item	elle admet une option suppl{\'e}mentaire <<~{\tt -f}~>>.
			Celle-ci permet de sp{\'e}cifier le nom d'un fichier contenant
			l'ensemble des expressions r{\'e}guli{\`e}res {\`a} rechercher dans
			les fichiers pr{\'e}sents sur la ligne de commande.
\end{itemize}

\begin{example}
\begin{verbatim}
egrep -f fic.expressions.regulieres fichier1 fichier2
\end{verbatim}
Cet exemple donne un apercu de l'utilisation de l'option <<~{\tt -f}~>>.
Toutes les lignes de <<~{\tt fichier1}~>> et <<~{\tt fichier2}~>>
satisfaisant les expressions r{\'e}guli{\`e}res contenues dans le fichier <<~{\tt
fic.expressions.regulieres}~>> seront affich{\'e}es {\`a} l'{\'e}cran.
\end{example}

\begin{example}
\begin{verbatim}
ypcat -k aliases | egrep '^schmoll:.*$' | cut -d@ -f2
\end{verbatim}
\vspace{2ex}
Cet exemple combine l'utilisation des filtres avec le symbole <<~\verb=|=~>>.
La commande <<~{\tt ypcat(1)}~>> permet d'interroger le syst{\`e}me d'annuaire
r{\'e}parti <<~{\sl NIS}\footnote{NIS = Network Information Service
\cite{nfs-nis-automnt}}~>>. <<~{\tt ypcat -k aliases}~>> interroge les
entr{\'e}es d{\'e}crivant les boites aux lettres du r{\'e}seau. Les r{\'e}ponses sont envoy{\'e}es
sur la sortie standard et leur format est~:\\[1ex]
\centerline{{\sl nom-utilisateur}{\tt :}{\sl boite aux lettres}}

\vspace{2ex}
La description des boites aux lettres ob{\'e}it {\`a} la syntaxe des adresses
de courrier {\'e}lectronique SMTP\footnote{SMTP = Simple Mail Transfert
Protocol.}~:\\[1ex]
\centerline{{\sl nom}{\tt @}{\sl domaine}}

\vspace{2ex}
La commande <<~{\tt egrep}~>> r{\'e}cup{\`e}re la sortie standard de la commande
<<~{\tt ypcat}~>> sur son entr{\'e}e standard et recherche les lignes
satisfaisant l'expression r{\'e}guli{\`e}re <<~\verb=^schmoll:.*$=~>>. Celle-ci
correspond {\`a} toute chaine commencant par <<~{\tt schmoll:}~>> suivit
d'un nombre quelconque de n'importe quels caract{\`e}res.

Le r{\'e}sultat de cette extraction sera donc toutes les lignes de la
forme~:\\[1ex]
\centerline{{\tt schmoll: {\sl nom}@{\sl domaine}}}

\vspace{2ex}
Ce r{\'e}sultat est redirig{\'e} sur l'entr{\'e}e standard de la commande <<~{\tt cut}~>>,
qui devra extraire le second champ de chaque ligne, le caract{\`e}re d{\'e}limiteur
{\'e}tant <<~{\tt @}~>>.

Nous aurons donc, comme r{\'e}sultat, tous les domaines des adresses
{\'e}lectroniques des boites aux lettres de <<~{\tt schmoll}~>>.
\end{example}

%%%%%%%%%%%%
\section{\texorpdfstring{Utilisation de <<~{\tt fgrep}~>>}{Utilisation de <<~fgrep~>>}}

\index{fgrep@\texttt{fgrep}}\index{grep@\texttt{grep}!fgrep@\texttt{fgrep}}La
commande <<~{\tt fgrep}~>> fonctionne comme la commande <<~{\tt grep}~>>
sauf qu'elle n'admet aucune expression r{\'e}guli{\`e}re. Par contre elle dispose
de l'option <<~{\tt -f}~>> identique {\`a} la commande <<~{\tt egrep}~>>.

Le fichier contiendra alors les cha{\^\i}nes de caract{\`e}res {\`a} rechercher sur
les fichiers en entr{\'e}e.

\begin{example}
\begin{verbatim}
fgrep "^[a-zA-Z]" fichier
\end{verbatim}
Cette commande permet de recherche la cha{\^\i}ne <<~\verb=[a-zA-Z]=~>>
tr{\`e}s exactement sans chercher {\`a} la traiter comme une expression r{\'e}guli{\`e}re.
\end{example}

\begin{example}
\begin{verbatim}
fgrep -f liste fichier1 fichier2
\end{verbatim}
Cet exemple donne un apercu de l'utilisation de l'option <<~{\tt -f}~>>.
Toutes les lignes dans <<~{\tt fichier1}~>> et <<~{\tt fichier2}~>>
contenant les chaines pr{\'e}sentes dans le fichier <<~{\tt liste}~>> seront
affich{\'e}es {\`a} l'{\'e}cran.
\end{example}

\begin{remarque}
La commande <<~{\tt fgrep}~>> poss{\`e}de un algorithme plus rapide que les
commandes <<~{\tt grep}~>> et <<~{\tt egrep}~>>. Elle permet de travailler
sur de tr{\`e}s gros volumes de donn{\'e}es.
\end{remarque}


%%%%%%%%%%%%%%%%%%%%%%%%%%%%%%%%%%%%%%%%%%%%%%%%%%%%%%%%%%%%
\section{\texorpdfstring{\label{adv-filters-f}Remarque sur l'utilisation de l'option <<~{\tt -f}~>>}{Remarque sur l'utilisation de l'option <<~-f~>>}}

{\`A} la r{\`e}gle 10 explicit{\'e}e {\`a} la section \ref{pgm-intro-basic-rules}, nous avions pr{\'e}ciser que la premi{\`e}re ligne d'un script devait sp{\'e}cifier le nom de l'ex{\'e}cutable associ{\'e} au shell utiliser.

Nous avons vu de m{\^e}me, que les commandes <<~{\tt egrep}~>> et <<~{\tt fgrep}~>> disposent de l'option <<~{\tt -f}~>>, permettant de pr{\'e}ciser un fichier de requ{\^e}tes de recherche.

Si l'analogie est faite avec le shell, le processus charg{\'e} d'{\'e}valuer chaque ligne du fichier est l'ex{\'e}cutable sp{\'e}cifi{\'e} au niveau de la premi{\`e}re ligne du fichier contenant les instructions. Par cons{\'e}quent si le fichier contenant les requ{\^e}tes de recherche~:
\begin{itemize}
	\item	a comme premi{\`e}re ligne <<~\verb=#!/usr/bin/egrep=~>> ou
			<<~\verb=#!/usr/bin/fgrep=~>>\footnote{Nous
			consid{\'e}rons ici que les ex{\'e}cutables de <<~{\tt egrep}~>> et
			<<~{\tt fgrep}~>> se trouvent dans
			le r{\'e}pertoire <<~{\tt /usr/bin}~>>. Pour en conna{\^\i}tre la
			localisation exacte sur votre syst{\`e}me, vous pouvez utiliser
			la commande <<~{\tt which(1)}~>> ou <<~{\tt whereis(1)}~>>.},
	\item	est accessible en ex{\'e}cution,
\end{itemize}
il peut {\^e}tre consid{\'e}r{\'e} comme un filtre ex{\'e}cutant les requ{\^e}tes de recherche associ{\'e} {\`a} <<~{\tt egrep}~>> ou <<~{\tt fgrep}~>> qu'il contient.

Si nous prenons tout d'abord l'exemple de <<~{\tt egrep}~>>, l'exemple suivant
permet de rechercher, sur l'entr{\'e}e standard (ou bien dans la liste de fichiers sp{\'e}cifi{\'e}e en argument), toutes les lignes satisfaisant les expressions r{\'e}guli{\`e}res suivantes~:
\begin{itemize}
	\item	<<~\verb=^[A-Z]=~>>, c'est-{\`a}-dire tout ligne commen\c{c}ant par une lettre majuscule,
	\item	<<~\verb=[0-9]$=~>>, c'est-{\`a}-dire toute ligne se terminant par un chiffre,
\end{itemize}
et d'afficher le r{\'e}sultat sur la sortie standard.
\begin{quote}
\begin{verbatim}
#!/usr/bin/egrep
^[A-Z]
[0-9]$
\end{verbatim}
\end{quote}

Le r{\'e}sultat affich{\'e} seront toutes les lignes satisfaisant au moins l'un des deux crit{\`e}res. Si cet exemple est enregistr{\'e} dans le fichier <<~{\tt localise}~>>, et que celui-ci est accessible en ex{\'e}cution, il suffira de taper la commande suivante pour obtenir le r{\'e}sultat escompt{\'e}~:
\begin{quote}
{\tt localise~{\sl fichier $\cdots$}}
\end{quote}
ou bien, si la commande est utilis{\'e}e comme un filtre~:
\begin{quote}
{\sl commande\_envoyant\_sur\_la\_sortie\_standard}~\verb=| localise=~{\sl fichier $\cdots$}
\end{quote}

Le principe d'utilisation pour la commande <<~{\tt fgrep}~>> reste identique {\`a} la restriction pr{\`e}s que <<~{\tt fgrep}~>> recherche des cha{\^\i}nes sans aucune interpr{\'e}tation des expressions sp{\'e}cifi{\'e}es dans le fichier. Par cons{\'e}quent, si nous avons~:
\begin{quote}
\begin{verbatim}
#!/usr/bin/fgrep
^[A-Z]
[0-9]$
\end{verbatim}
\end{quote}
cette nouvelle commande, en supposant que le fichier soit enregistr{\'e} sous le nom <<~{\tt localise2}~>>, recherchera sur les fichiers sp{\'e}cifi{\'e}s en argument, ou bien sur l'entr{\'e}e standard, toutes les lignes contenant au moins l'une des deux chaines suivantes~:
\begin{itemize}
	\item	<<~\verb=^[A-Z]=~>>,
	\item	<<~\verb=[0-9]$=~>>.
\end{itemize}

Tout comme la commande pr{\'e}c{\'e}dente, l'appel pourra se faire selon les deux m{\'e}thodes ci-apr{\`e}s~:
\begin{quote}
{\tt localise2~{\sl fichier $\cdots$}}
\end{quote}
ou bien, si la commande est utilis{\'e}e comme un filtre~:
\begin{quote}
{\sl commande\_envoyant\_sur\_la\_sortie\_standard}~\verb=| localise2=~{\sl fichier $\cdots$}
\end{quote}

