%%%%%%%%%%%%%%%%%%%%%%%%%%%%%%%%%%%%%%%%%%%%%%%%%%%%%%%%%%%%%%%%%%%%%%%%
%                                                                      %
% This program is free software; you can redistribute it and/or modify %
% it under the terms of the GNU General Public License as published by %
% the Free Software Foundation; either version 2 of the License, or    %
% (at your option) any later version.                                  %
%                                                                      %
% This program is distributed in the hope that it will be useful,      %
% but WITHOUT ANY WARRANTY; without even the implied warranty of       %
% MERCHANTABILITY or FITNESS FOR A PARTICULAR PURPOSE.  See the        %
% GNU General Public License for more details.                         %
%                                                                      %
% You should have received a copy of the GNU General Public License    %
% along with this program; if not, write to the Free Software          %
% Foundation, Inc., 51 Franklin St, Fifth Floor, Boston,               %
% MA  02110-1301  USA                                                  %
%                                                                      %
%%%%%%%%%%%%%%%%%%%%%%%%%%%%%%%%%%%%%%%%%%%%%%%%%%%%%%%%%%%%%%%%%%%%%%%%
%
%	$Id$
%

\setcounter{remarque-cnt}{1}
\setcounter{example-cnt}{1}
\chapter{\label{adv-fltrs-awk}Utilisation de "{\tt awk}"}
\thispagestyle{fancy}

%%%%%%%%%%%%%%%%%%%%%%%%%%%%%%%%%%%%%%%%%%%%%%%%%%%%%%%%%%%%
\section{Introduction}

\index{awk@\texttt{awk}}"\texttt{awk}" est un processeur d'{\'e}l{\'e}ments syntaxiques qui~:
\begin{itemize}
	\item	sert {\`a} traiter, de mani{\`e}re non interactive, un texte au
			niveau de ses {\'e}l{\'e}ments syntaxiques,
	\item	travaille sur la base d'une ligne, et ligne apr{\`e}s ligne,
	\item	permet de d{\'e}finir, analyser, transformer les mots ou {\'e}l{\'e}ments
			syntaxiques qui composent chaque ligne.
\end{itemize}

"\texttt{awk}" traitera chaque ligne en entr{\'e}e comme un
\index{awk@\texttt{awk}!enregistrement}enregistrement. C'est ce terme qui sera
employ{\'e} par la suite. Chaque enregistrement est compos{\'e} de champs. Ces
\index{awk@\texttt{awk}!champ}champs sont s{\'e}par{\'e}s, par d{\'e}faut, par un ou plusieurs espaces ou bien une ou plusieurs tabulations. C'est aussi un langage de programmation dont la fonction premi{\`e}re est de rechercher des cha{\^\i}nes de caract{\`e}res suivant certains crit{\`e}res et d'y appliquer des actions. On aura donc toujours un mod{\`e}le~: \centerline{\textsl{s{\'e}lection} +
\textsl{action}}

Le corps de chaque action est un bloc constitu{\'e} d'une ou plusieurs commandes, d{\'e}limit{\'e} par les caract{\`e}res "\verb={=" et <<\verb=}=". On aura alors~:
\begin{center}
{\tt s{\'e}lection \{ commande }$\cdots$\verb= }=
\end{center}

Pour la petite histoire, le nom "\texttt{awk}" est d{\'e}riv{\'e} des initiales de ses auteurs: Alfred V. \textsc{Aho}, Peter J. \textsc{Weinberger} et Brian W.
\textsc{Kerninghan}.

\begin{definition}{Syntaxe}
\texttt{awk '}\textsl{corps du programme awk}\verb= ' [=\textsl{fichier}=$\cdots$\verb= ]=\\
ou \texttt{awk -f }\textsl{fichier.programme}\verb= [=\textsl{fichier} $\cdots$\verb=]=
\end{definition}

Dans le premier cas de syntaxe, le corps du programme "\texttt{awk}" est appliqu{\'e} directement aux fichiers ou {\`a} l'entr{\'e}e standard. Dans le second cas de syntaxe, le programme "\texttt{awk}" est contenu dans un fichier et appliqu{\'e} aux fichiers ou {\`a} l'entr{\'e}e standard.

%%%%%%%%%%%%%%%%%%%%%%%%%%%%%%%%%%%%%%%%%%%%%%%%%%%%%%%%%%%%
\section{Les s{\'e}lecteurs}

%%%%%%%%%%%%
\subsection{Introduction, D{\'e}finition}

On a vu que chaque requ{\^e}te "\texttt{awk}" {\'e}tait constitu{\'e}e d'une s{\'e}lection et d'une action. Nous allons examiner dans ce paragraphe la syntaxe pour la sp{\'e}cification des s{\'e}lections. Les s{\'e}lections des lignes en entr{\'e}es sont d{\'e}crites par des \index{awk@\texttt{awk}!s{\'e}lecteur}s{\'e}lecteurs.

Un s{\'e}lecteur est~:
\begin{itemize}
	\item	une expression r{\'e}guli{\`e}re,
	\item	un mot clef.
\end{itemize}

Chaque fois que l'expression d{\'e}crite par un s{\'e}lecteur est vraie, l'action cor\-res\-pon\-dante est ex{\'e}cut{\'e}e. Les s{\'e}lecteurs et les actions peuvent faire appel {\`a} des variables, {\`a} des param{\`e}tres et {\`a} des expressions logiques. Les actions, elles, peuvent contenir des structures de contr{\^o}le ("\texttt{if}", "\texttt{while}", etc.).

%%%%%%%%%%%%
\subsection{Les s{\'e}lecteurs pr{\'e}d{\'e}fini}

Les deux seuls s{\'e}lecteurs pr{\'e}d{\'e}finis dans "\texttt{awk}" sont~:
\begin{itemize}
	\item	\index{awk@\texttt{awk}!s{\'e}lecteur!BEGIN@\texttt{BEGIN}}\texttt{BEGIN}
	\item	\index{awk@\texttt{awk}!s{\'e}lecteur!END@\texttt{BEGIN}}\texttt{END}
\end{itemize}

Le s{\'e}lecteur "\texttt{BEGIN}" est vrai avant le traitement du premier
enregistrement. Ce s{\'e}lecteur est g{\'e}n{\'e}ralement utilis{\'e} pour initialiser des variables, d{\'e}finir des champs de saisie, etc.

Le s{\'e}lecteur "\texttt{END}" est vrai apr{\`e}s le traitement du dernier enregistrement. Ce s{\'e}lecteur est g{\'e}n{\'e}ralement utilis{\'e} pour afficher les r{\'e}sultats finaux.

\begin{example}
L'exemple suivant permet d'afficher les noms de tous les fichiers du r{\'e}pertoire et leurs tailles.
\begin{verbatim}
ls -al | awk '
    BEGIN {
        totalsize = 0
        print "Fichier(Taille) du r{\'e}pertoire courant."
    }
    {
        totalsize += $5
        printf ("%s (%d)\n", $9, $5)
    }
    END {
        printf (Taille totale = %d octet(s).\n", totalsize)
        print "Fin de la liste."
    }
'
\end{verbatim}

Pour rappel, la commande "\texttt{ls -al}" permet d'obtenir la liste de
tous les fichiers dans le r{\'e}pertoire courant, y compris les fichiers
cach{\'e}s. Le cinqui{\`e}me champ correspond {\`a} la taille du fichier, le neuvi{\`e}me
{\`a} son nom (cf. section \ref{cmds-unix-ls}).

Lors de l'ex{\'e}cution du s{\'e}lecteur "\texttt{BEGIN}", le programme
"\texttt{awk}" initialise la variable "\texttt{totalsize}" {\`a} $0$,
afin de pouvoir effectuer un cumul sur le cinqui{\`e}me champ, c'est-{\`a}-dire
la taille des fichiers. Il affiche aussi un message avant ex{\'e}cution.

Pour chaque enregistrement, c'est-{\`a}-dire pour chaque fichier,
on cumule sur le cinqui{\`e}me champ ("\verb,totalsize += $5,")
et on affiche les informations d{\'e}sir{\'e}es.

Enfin, lors de l'ex{\'e}cution du s{\'e}lecteur "\texttt{END}", le
programme "\texttt{awk}" affiche la valeur du cumul (variable
"\texttt{totalsize}") et un message de fin.

Par exemple, si la commande "\texttt{ls -al}" produit le r{\'e}sultat
suivant~:
\begin{verbatim}
-rw-r----- 1 schmoll esme     102 Jul 12 10:00 .cshrc
-rw-r----- 1 schmoll esme     123 Jul 12 10:00 .login
-rw-r----- 1 schmoll esme     124 Jul 12 10:00 .logout
-rw-r----- 1 schmoll esme      24 Jul 25 10:00 mes
-rw-r--r-- 1 schmoll esme      12 Jul 25 12:00 fichiers
-rwxr-xr-x 1 schmoll esme    1337 Jul 25 12:00 a
-rw-r--r-- 1 schmoll esme 1221337 Jul 25 12:00 moi
\end{verbatim}
le programme "\texttt{awk}" produira la sortie suivante~:
\begin{verbatim}
Fichier(Taille) du r{\'e}pertoire courant.
.cshrc (102)
.login (123)
.logout (124)
mes (24)
fichiers (12)
a (1337)
moi (1221337)
Taille totale = 1223058 octet(s).
\end{verbatim}
\end{example}


%%%%%%%%%%%%
\subsection{Les r{\`e}gles de s{\'e}lection}

Un s{\'e}lecteur est une expression r{\'e}guli{\`e}re  qui va {\^e}tre compar{\'e}e {\`a}
l'enregistrement courant du fichier en entr{\'e}e. Si une correspondance est
trouv{\'e}e entre l'expression r{\'e}guli{\`e}re et l'enregistrement, le s{\'e}lecteur
devient vrai et l'action correspondante est ex{\'e}cut{\'e}e. Cette expression
r{\'e}guli{\`e}re peut aussi s'appliquer seulement sur un ou plusieurs champs.

%%%%%%%%%%%%
\subsection{Les caract{\`e}res sp{\'e}ciaux pour la s{\'e}lection}

\begin{longtable}{|l|p{6cm}|}
	\hline
	\multicolumn{2}{|r|}{Suite de la page pr{\'e}c{\'e}dente $\cdots$}	\\
	\hline
	Caract{\`e}re	&	Description	\\
	\hline
\endhead
	\hline
	Caract{\`e}re	&	Description	\\
	\hline
\endfirsthead
	\hline
	\multicolumn{2}{|r|}{Suite page suivante $\cdots$}	\\
	\hline
\endfoot
	\hline
\endlastfoot
	\hline
		\verb=/=		&
			D{\'e}limiteur d'une expression r{\'e}guli{\`e}re.					\\[1ex]
		\verb="=		&
			D{\'e}limiteur d'une cha{\^\i}ne de caract{\`e}res.					\\[1ex]
		\verb=\=		&
			Permet de sp{\'e}cifier un caract{\`e}re sp{\'e}cial en tant que caract{\`e}re
			normal. Par exemple "\verb=\/=" d{\'e}signe le caract{\`e}re "\verb=/=".
			\\[1ex]
		\verb=$=$n$		&
			Repr{\'e}sente le $n^{i\grave{e}me}$ champ de l'enregistrement courant.
			La valeur de "$n$" n'est limit{\'e}e que par le nombre de champs dans
			l'enregistrement courant.								\\[1ex]
		\verb=$0=		&
			Repr{\'e}sente la totalit{\'e} de l'enregistrement en entr{\'e}e.	\\[1ex]
\end{longtable}

%%%%%%%%%%%%
\subsection{Les expressions logiques pour la s{\'e}lection}

\begin{longtable}{|@{\hspace{0.5ex}}l@{\hspace{0.5ex}}|@{\hspace{0.5ex}}p{6cm}@{\hspace{0.5ex}}|}

	\hline
	\multicolumn{2}{|r|}{Suite de la page pr{\'e}c{\'e}dente $\cdots$}	\\
	\hline
	Op{\'e}rateur	&	Description	\\
	\hline
\endhead
	\hline
	Op{\'e}rateur	&	Description	\\
	\hline
\endfirsthead
	\hline
	\multicolumn{2}{|r|}{Suite page suivante $\cdots$}	\\
	\hline
\endfoot
	\hline
\endlastfoot
	\hline
		\verb=<=	&	Inf{\'e}rieur {\`a}			\\[1ex]
		\verb=>=	&	Sup{\'e}rieur {\`a}			\\[1ex]
		\verb,==,	&	{\'E}galit{\'e}				\\[1ex]
		\verb,!=,	&	Diff{\'e}rent			\\[1ex]
		\verb=&&=	&	\textsl{ET} logique	\\[1ex]
		\verb=||=	&	\textsl{OU} logique	\\[1ex]
		\verb=~=	&	Permet de comparer l'expression r{\'e}guli{\`e}re
						{\`a} un champ pr{\'e}cis.	\\[1ex]
\end{longtable}

%%%%%%%%%%%%
\subsection{Syntaxe des s{\'e}lecteurs}

La syntaxe des \index{awk@\texttt{awk}!s{\'e}lecteur!syntaxe}s{\'e}lecteur peut s'exprimer de trois fa\c{c}ons~:
\begin{itemize}
	\item	s{\'e}lection en fonction d'une expression r{\'e}guli{\`e}re,
	\item	s{\'e}lection en fonction d'expression logiques,
	\item	s{\'e}lection en utilisant les deux formats pr{\'e}c{\'e}dents.
\end{itemize}

Dans le premier cas de figure, si aucun champ n'est sp{\'e}cifi{\'e} dans le s{\'e}lecteur,
l'expression r{\'e}guli{\`e}re s'applique {\`a} l'ensemble de l'enregistrement. On aura donc
les syntaxes suivantes~:
\begin{quote}
\verb=/=expression$_{r\acute{e}guli\grave{e}re}$\verb=/=\\
\verb=$=champ~\verb=~ /=expression$_{r\acute{e}guli\grave{e}re}$\verb=/=
\end{quote}

Dans le second cas de figure, on peut {\'e}crire une expression logique en utilisant
les op{\'e}rateurs logiques pr{\'e}c{\'e}demment d{\'e}crits. On pourra donc avoir, par exemple~:
\begin{quote}
\begin{verbatim}
$1 == $2
$2 < $3
$2 != $3
( $2 < $3 ) && ($2 > $4)
(($2 > 100) || ( $2 == $3*50)) && ($4 > 10)
\end{verbatim}
\end{quote}

Il est {\'e}videmment possible d'utiliser les op{\'e}rateurs logiques pour relier les
expressions correspondant au premier cas de figure. On pourra donc avoir~:
\begin{quote}
\begin{verbatim}
($1 ~ /[a-z]/) && ($2 ~ /[0-9]/)
($1 ~ /[a-z]/) && ($2 ~ /[0-9]/) && ($2 < 10)
\end{verbatim}
\end{quote}

%%%%%%%%%%%%%%%%%%%%%%%%%%%%%%%%%%%%%%%%%%%%%%%%%%%%%%%%%%%%
\section{Les variables}

\index{awk@\texttt{awk}!variables}"\texttt{awk}" permet l'utilisation de
variables internes. Ces variables peuvent {\^e}tre de type num{\'e}rique ou bien
alphanum{\'e}rique.

Il poss{\`e}de aussi un certain nombre de variables pr{\'e} d{\'e}finies permettant de
param{\`e}trer l'environnement de "\texttt{awk}" ou bien d'obtenir des informations
sur le contexte courant. Ces \index{awk@\texttt{awk}!variables pr{\'e} d{\'e}finies}variables sont~:

\begin{tabular}{|l|p{10cm}|}
	\hline
		\multicolumn{1}{|c|}{Variable}		&
		\multicolumn{1}{|c|}{Description}	\\
	\hline \hline
		\index{awk@\texttt{awk}!variables!FILENAME@\texttt{FILENAME}}\texttt{FILENAME}	&
		nom du fichier courant en entr{\'e}e.	\\
	\hline
		\index{awk@\texttt{awk}!variables!NR@\texttt{NR}}\texttt{NR}					&
		num{\'e}ro de l'enregistrement courant.	\\
	\hline
		\index{awk@\texttt{awk}!variables!NF@\texttt{NF}}\texttt{NF}					&
		nombre de champs dans l'enregistrement courant.	\\
	\hline
		\index{awk@\texttt{awk}!variables!FS@\texttt{FS}}\texttt{FS}					&
		caract{\`e}re s{\'e}parateur de champs.		\\
	\hline
		\index{awk@\texttt{awk}!variables!RS@\texttt{RS}}\texttt{RS}					&
		caract{\`e}re de s{\'e}paration d'enregistrements.	\\
	\hline
		\index{awk@\texttt{awk}!variables!\$0@\texttt{\$0}}\verb=$0=						&
		l'enregistrement courant.			\\
	\hline
		\index{awk@\texttt{awk}!variables!\$i@\texttt{\$}\textit{i}}\verb=$=\textit{n}	&
		le $n^{i\grave{e}me}$ champ de l'enregistrement courant.	\\
	\hline
		\index{awk@\texttt{awk}!variables!IFS@\texttt{IFS}}\texttt{IFS}					&
		caract{\`e}re s{\'e}parateur de champs en entr{\'e}e.	\\
	\hline
		\index{awk@\texttt{awk}!variables!IRS@\texttt{IRS}}\texttt{IRS}					&
		caract{\`e}re de s{\'e}paration d'enregistrements en entr{\'e}e.	\\
	\hline
		\index{awk@\texttt{awk}!variables!OFS@\texttt{OFS}}\texttt{OFS}					&
		caract{\`e}re s{\'e}parateur de champs en sortie.	\\
	\hline
		\index{awk@\texttt{awk}!variables!ORS@\texttt{ORS}}\texttt{ORS}					&
		caract{\`e}re de s{\'e}paration d'enregistrements en sortie.	\\
	\hline
		\index{awk@\texttt{awk}!variables!OFMT@\texttt{OFMT}}\texttt{OFMT}				&
		format de sortie pour les chiffres.	\\
	\hline
\end{tabular}

\begin{example}
Commande~:
\begin{quote}
\begin{verbatim}
awk '
    BEGIN {
        print "Liste du fichier ", FILENAME
        print
        printf ("N.L.\tUID"\t|\tNom de login\n")
        FS=":"
        OFS="\t|\t"
    }
    {
        printf ("%03d:\t",NR)
        print $3,$1
    }
' /etc/passwd
\end{verbatim}
\end{quote}

Le programme "\texttt{awk}" pr{\'e}c{\'e}dent s'applique au fichier "\texttt{/etc/passwd}",
contenant la liste des utilisateurs autoris{\'e}s {\`a} se connecter au syst{\`e}me. Ce fichier
se compose des champs suivants s{\'e}par{\'e}s par le caract{\`e}re "\texttt{:}"~:
\begin{itemize}
	\item	le nom de connexion ou "\textsl{logname}",
	\item	le mot de passe crypt{\'e} de l'utilisateur,
	\item	son identifiant num{\'e}rique unique, l'UID,
	\item	son identifiant num{\'e}rique de groupe, le GID,
	\item	un champ de commentaires, appel{\'e} "GCOS", contenant usuellement
			le pr{\'e}nom et le nom complet de l'utilisateur,
	\item	son r{\'e}pertoire de connexion, c'est-{\`a}-dire le r{\'e}pertoire qu'il aura
			par d{\'e}faut lorsqu'il se connectera au syst{\`e}me\footnote{Ce r{\'e}pertoire
			est celui utilis{\'e} par d{\'e}faut par la commande "\texttt{cd}", cf.
			section \ref{cmds-pwd-cd}.}
	\item	le nom de l'ex{\'e}cutable correspondant {\`a} son interpr{\'e}teur de commandes.
\end{itemize}
Pour plus de renseignements sur ce fichiers, reportez-vous {\`a} "\texttt{passwd(5)}".

Ce programme se d{\'e}compose en deux parties~:
\begin{enumerate}
	\item	Le s{\'e}lecteur "\texttt{BEGIN}" permet d'afficher un texte pr{\'e}limaire
			indiquant~:
			\begin{itemize}
				\item	la liste des utilisateurs du fichier trait{\'e} (r{\'e}f{\'e}renc{\'e} par la
						variable "\texttt{FILENAME}",
				\item	une ligne blanche (commande "\texttt{print}" sans arguments),
				\item	la premi{\`e}re ligne du tableau g{\'e}n{\'e}r{\'e} par le programme.
			\end{itemize}
			L'initialisation de la variable "\texttt{FS}" permet de sp{\'e}cifier la nouvelle
			valeur du s{\'e}parateur de champ. Celui-ci devient le caract{\`e}re "\texttt{:}".
			L'initialisation de la variable "\texttt{OFS}" permet de sp{\'e}cifier
			le nouveau format d'affichage de la commande "\texttt{print}". Chaque
			argument de cette commande sera s{\'e}par{\'e} par le caract{\`e}re "\verb=|="
			pr{\'e}c{\'e}d{\'e} et suivi d'une tabulation.\\[0.5cm]
	\item	Le second s{\'e}lecteur ne poss{\`e}de aucun crit{\`e}re, les actions qui y sont pr{\'e}sentes
			s'appliquent {\`a} chaque enregistrement du fichier c'est-{\`a}-dire {\`a} chaque
			ligne. Les deux actions pr{\'e}sentes ont les fonctions suivantes~:
			\begin{itemize}
				\item	la premi{\`e}re affiche le num{\'e}ro de l'enregistrement courant sur
						trois chiffres\footnote{Les formats utilis{\'e}s
						par la commande "\texttt{printf}" de "\texttt{awk}" sont
						identiques {\`a} ceux de la fonction "\texttt{printf(3)}".}
						gr{\^a}ce {\`a} la variable "\texttt{NR}",
				\item	la seconde affiche le troisi{\`e}me et le premier champ en utilisant
						le format de sortie contenu dans la variable "\texttt{OFS}".
			\end{itemize}
\end{enumerate}

Le r{\'e}sultat obtenu sera donc~:
\begin{quote}
\begin{verbatim}
liste du fichier /etc/passwd

N.L.	UID	|	Nom de login
001:	0	|	root
\end{verbatim}
etc.
\end{quote}
\end{example}


La d{\'e}finition des variables utilisateur est dynamique. Par cons{\'e}quent, leur
d{\'e}claration se fait {\`a} la premi{\`e}re utilisation et leur type s'adapte en fonction
de la valeur {\`a} laquelle elle a {\'e}t{\'e} initialis{\'e}e. Le type d'une variable d{\'e}pend de
son utilisation. Il peut {\^e}tre~:
\begin{itemize}
	\item	num{\'e}rique (virgule flottante ou enti{\`e}re)
	\item	alphanum{\'e}rique
\end{itemize}

Les variables sont d{\'e}clar{\'e}es de fa\c{c}on globale {\`a} tous les blocs action de toutes
les requ{\^e}tes "\texttt{awk}". Par cons{\'e}quent, une variable cr{\'e}{\'e}e dans la requ{\^e}te
"\texttt{BEGIN}", sera accessible par toutes les requ{\^e}tes "\texttt{awk}".


%%%%%%%%%%%%
\subsection{Les tableaux}

\index{awk@\texttt{awk}!tableaux}"\texttt{awk}" sait aussi g{\'e}rer des
tableaux. Les types disponibles pour ces tableaux sont les m{\^e}mes que les
variables. On a donc les types suivants~:
\begin{itemize}
	\item	type num{\'e}rique (virgule flottante ou entier),
	\item	alphanum{\'e}rique.
\end{itemize}

La dimension d'un tableau est dynamique. Il n'y a donc pas besoin de d{\'e}clarer
sa dimension  avant de l'utiliser, il suffit de l'initialiser. Les indices
peuvent {\^e}tre num{\'e}riques ou bien alphanum{\'e}riques, c'est-{\`a}-dire
qu'un {\'e}l{\'e}ment du tableau peut {\^e}tre repr{\'e}sent{\'e} par un indice
num{\'e}rique ou bien par un indice symbolis{\'e} par une cha{\^\i}ne de
caract{\`e}res. Ce deuxi{\`e}me cas correspond aux tableaux
"\textsl{associatifs}". Tout comme les variables, les tableaux sont communs
{\`a} tous les blocs "\textsl{action}" de toutes les requ{\^e}tes
"\texttt{awk}"~: ils sont d{\'e}clar{\'e}s "\textsl{en global}".

\begin{example}
Indi\c{c}age de tableaux~:
\begin{quote}
\begin{verbatim}
array[0]="ABC"
other_array[schmoll]="bidule"
\end{verbatim}
\end{quote}
\end{example}

%%%%%%%%%%%%%%%%%%%%%%%%%%%%%%%%%%%%%%%%%%%%%%%%%%%%%%%%%%%%
\section{Les actions}

Les \index{awk@\texttt{awk}!action}actions sont des blocs constituant une
requ{\^e}te "\texttt{awk}". Ils d{\'e}crivent les op{\'e}rations {\`a}
effectuer lorsque la s{\'e}lection d{\'e}crite en t{\^e}te de requ{\^e}te est
v{\'e}rifi{\'e}e. Les gens connaissant le langage C retrouveront la plupart des
fonctions standards avec des syntaxes identiques. On trouvera aussi un ensemble
de fonctions sp{\'e}cifiques.

Les \index{awk@\texttt{awk}!op{\'e}rateurs arithm{\'e}tiques}op{\'e}rateurs
arithm{\'e}tiques autoris{\'e}s dans ces blocs sont explicit{\'e}s dans le
tableau suivant~:

\begin{longtable}{|l|p{10cm}|}
	\hline
		\multicolumn{2}{|r|}{Suite de la page pr{\'e}c{\'e}dente $\cdots$}	\\
	\hline
		\multicolumn{1}{|c|}{Op{\'e}rateur}		&
		\multicolumn{1}{|c|}{Description}	\\
	\hline
\endhead
	\hline
		\multicolumn{1}{|c|}{Op{\'e}rateur}		&
		\multicolumn{1}{|c|}{Description}	\\
	\hline
\endfirsthead
	\hline
		\multicolumn{2}{|r|}{Suite page suivante $\cdots$}	\\
	\hline
\endfoot
	\hline
\endlastfoot
	\hline
		\texttt{=}	&	affectation						\\
	\hline
		\texttt{+}	&	addition binaire				\\
	\hline
		\texttt{++}	&	incr{\'e}mentation de $1$		\\
	\hline
		\texttt{+=}	&	addition unaire					\\
	\hline
		\texttt{-}	&	soustraction binaire			\\
	\hline
		\verb=--=	&	d{\'e}cr{\'e}mentation de $1$	\\
	\hline
		\texttt{-=}	&	soustraction unaire				\\
	\hline
		\verb=*=	&	multiplication binaire			\\
	\hline
		\verb,*=,	&	multiplication unaire			\\
	\hline
		\verb=/=	&	division binaire				\\
	\hline
		\verb,/=,	&	division unaire					\\
	\hline
		\verb=%=	&	modulo							\\
\end{longtable}



La concat{\'e}nation de champs se fait sans op{\'e}rateur sp{\'e}cifique. Il suffit
de lister les cha{\^\i}nes {\`a} concat{\'e}ner.

\begin{example}
\begin{tabular}{l@{\hspace{1cm}}p{6cm}}
	\verb,cumul=0,						&	\\
	\verb,taux=$2,						&	\\
	\verb,nom=$1 "du genoux" $2,		&	\\
	\verb,nom=$1 $2,					&	\\
	\verb,cumul=cumul + 10,				&	\\
	\verb,ttc=$4 + $5,					&	\\
	\verb,cumul ++,						&	identique {\`a}	"\texttt{cumul = cumul + 1}"\\
	\verb,cumul += $2,					&	identique {\`a}	"\texttt{cumul = cumul + \$2}"\\
	\verb,cumul = cumul - $2,			&	\\
	\verb,ht = ttc - tva,				&	\\
	\verb,cumul --,						&	identique {\`a}	"\texttt{cumul = cumul - 1}"\\
	\verb,cumul -= $2,					&	identique {\`a}	"\texttt{cumul = cumul - \$2}"\\
	\verb,total = cumul * 100,			&	\\
	\verb,carre = $1 * $1,				&	\\
	\verb,cumul *= $2,					&	identique {\`a}	"\texttt{cumul = cumul * \$2}"\\
	\verb,pourcent = $2 / cumul * 100,	&	\\
	\verb,cumul /= $2,					&	identique {\`a}	"\texttt{cumul = cumul / \$2}"\\
	\verb,reste = cumul % 3,			&	\\
\end{tabular}
\end{example}

%%%%%%%%%%%%
\subsection{Les fonctions pr{\'e}d{\'e}finies}

La liste des \index{awk@\texttt{awk}!fonctions pr{\'e}d{\'e}finies}fonctions
suivantes n'est pas exhaustive. Elle donne celles qui sont usuellement
utilis{\'e}es. Les tableaux explicitent ces fonctions. Le tableau
\ref{tab-awk-fct-num} donne une liste de fonctions acceptant un argument de type
num{\'e}rique, le tableau \ref{tab-awk-fct-str} donne une liste de fonctions
acceptant un argument de type "\textsl{cha{\^\i}ne de caract{\`e}res}".

\begin{table}[hbtp]
\begin{tabular}{|l|p{10cm}|}
	\hline
		\multicolumn{1}{|c|}{Fonction}			&
		\multicolumn{1}{|c|}{Description}		\\
	\hline \hline
		\index{awk@\texttt{awk}!fonctions pr{\'e}d{\'e}finies!sqrt@\texttt{sqrt}}\texttt{sqrt(\textsl{arg})}	&
			renvoie la racine carr{\'e} de l'argument.							\\
	\hline
		\index{awk@\texttt{awk}!fonctions pr{\'e}d{\'e}finies!log@\texttt{log}}\texttt{log(\textsl{arg})}		&
			renvoie le logarithme n{\'e}p{\'e}rien de l'argument.				\\
	\hline
		\index{awk@\texttt{awk}!fonctions pr{\'e}d{\'e}finies!exp@\texttt{exp}}\texttt{exp(\textsl{arg})}		&
			renvoie l'exponentiel de l'argument.								\\
	\hline
		\index{awk@\texttt{awk}!fonctions pr{\'e}d{\'e}finies!int@\texttt{int}}\texttt{int(\textsl{arg})}		&
			renvoie la partie enti{\`e}re\footnote{C'est-{\`a}-dire
			la fonction math{\'e}matique $E(x)$.} de l'argument..				\\
	\hline
\end{tabular}
\caption{\label{tab-awk-fct-num}Fonctions acceptant un argument de type num{\'e}rique.}
\end{table}

\begin{table}[hbtp]
\begin{tabular}{|l|p{8cm}|}
	\hline
		\multicolumn{1}{|c|}{Fonction}					&
		\multicolumn{1}{|c|}{Description}				\\
	\hline \hline
		\index{awk@\texttt{awk}!fonctions pr{\'e}d{\'e}finies!length@\texttt{length}}\texttt{length}\footnote{L'appel {\`a} "\texttt{length}" ne poss{\`e}de aucun argument.}
														&
		renvoie la longueur de l'enregistrement courant.		\\
	\hline
		\texttt{length(\textsl{arg})}							&
		renvoie la longueur de la chaine pass{\'e}e en argument.	\\
	\hline
		\index{awk@\texttt{awk}!fonctions pr{\'e}d{\'e}finies!substr@\texttt{substr}}\texttt{substr(\textsl{arg},\textsl{m}[,\textsl{n}])}		&
		renvoie la sous cha{\^\i}ne de la chaine "\textsl{arg}" commen\c{c}ant
		{\`a} la position "\textsl{m}" et de longueur "\textsl{n}". Le premier caract{\`e}re
		est {\`a} la position "$1$"\footnote{Alors qu'en langage C, le premier
		caract{\`e}re d'une chaine est {\`a} la position "$0$".}. Si "\textsl{n}" n'est pas
		sp{\'e}cifi{\'e}, la fonction "\texttt{substr}" renvoie la fin de l'argument {\`a} partir
		de la position "\textsl{m}".							\\
	\hline
		\index{awk@\texttt{awk}!fonctions pr{\'e}d{\'e}finies!index@\texttt{index}}\texttt{index($str_1$,$str_2$)}					&
		renvoie la position de "$str_2$" dans la cha{\^\i}ne "$str_1$". Si
		"$str_1$" ne contient pas la cha{\^\i}ne "$str_2$", "\texttt{index}"
		renvoie "$0$"\footnote{$0$ ne correspond {\`a} aucune position dans une chaine
		puisque, dans "\texttt{awk}", le premier caract{\`e}re d'une chaine a la position
		"$1$", \textbf{ce qui n'est pas le cas en langage C.}}.	\\
	\hline
		\index{awk@\texttt{awk}!fonctions pr{\'e}d{\'e}finies!print@\texttt{print}}\texttt{print [$arg_1$[,$arg_2$],$\cdots$] [> \textsl{dest}]}	&
		affiche les arguments "$arg_1$", "$arg_1$", $\cdots$ sur
		la sortie standard sans les formater. Avec l'option "\texttt{>~\textsl{dest}}",
		l'affichage est redirig{\'e} sur le fichier "\textsl{dest}" au lieu de la
		sortie standard.										\\
	\hline
		\index{awk@\texttt{awk}!fonctions pr{\'e}d{\'e}finies!printf@\texttt{printf}}\texttt{printf(\textsl{format},$arg_1$,$arg_2$,$\cdots$)	[> \textsl{dest}]}	&
		affiche les arguments $arg_1$, $arg_2$, $\cdots$ sur la sortie standard apr{\`e}s
		les avoir format{\'e}s {\`a} l'aide de la cha{\^\i}ne de contr{\^o}le "\textsl{format}". Avec
		l'option "\texttt{> \textsl{dest}}", l'affichage est redirig{\'e} sur le fichier
		"\textsl{dest}" au lieu de la sortie standard.			\\
	\hline
		\index{awk@\texttt{awk}!fonctions pr{\'e}d{\'e}finies!sprintf@\texttt{sprintf}}\texttt{sprintf(\textsl{format},$arg_1$,$arg_2$,$\cdots$)}	&
		renvoie une cha{\^\i}ne de caract{\`e}res format{\'e}e int{\'e}grant $arg_1$, $arg_2$, $\cdots$
		correspondant aux instruction de formatage en fonction de la
		cha{\^\i}ne de contr{\^o}le "\textsl{format}". \textbf{Attention}, cette fonction a un
		comportement diff{\'e}rent de celui de la fonction "\texttt{sprintf(3)}".	\\
	\hline
\end{tabular}
\caption{\label{tab-awk-fct-str}Fonctions acceptant des arguments de type alphanum{\'e}rique.}
\end{table}

\begin{remarque}
Pour plus de pr{\'e}cisions sur les instructions de formatage, reportez-vous
{\`a} l'annexe \ref{ann-format}.
\end{remarque}

\begin{example}
\noindent \textsl{Exemple d'utilisation des fonctions pr{\'e}d{\'e}finies~:}\\
\begin{verbatim}
if (length > 80 ) {
    print "la ligne no: ", NR, " du fichier ",FILENAME, \
        "est trop longue"
}
lentexte += length ($2)
val=sqrt (cumul)
val = log(sqrt(cumul))
val = exp(log(cumul))
modulo = int(cumul/100) * 100
codepostal = substr("75006 Paris",1,5)
pos = index("75006 Paris","Paris")      renvoie 7
print $1,$2,cumul
print "resultats", cumul > /tmp/result.tmp
print "cumuls", $1+$2 > $3
nom = sprintf ("%10.10s %10.10s .", nom, prenom)
\end{verbatim}
\end{example}

%%%%%%%%%%%%
\subsection{Les fonctions utilisateur}

En plus des fonctions pr{\'e}d{\'e}finies, l'utilisateur peut d{\'e}finir ses
propres \index{awk@\texttt{awk}!fonctions utilisateur}fonctions. Ces fonctions
peuvent se trouver n'importe o{\`u} dans le corps du programme
"\texttt{awk}". La d{\'e}claration d'une fonction se fait de la fa\c{c}on
suivante~:
\begin{quote}
\begin{verbatim}
function nom_fonction (arguments)
{
    instructions
}
\end{verbatim}
\end{quote}

La fonction peut {\^e}tre appel{\'e}e dans n'importe quel bloc action d'une
requ{\^e}te "\texttt{awk}". Il suffit de la r{\'e}f{\'e}rencer par son nom. Les
fonctions utilisateurs peuvent {\^e}tre r{\'e}cursives.

\begin{example}
\begin{verbatim}
function factoriel (num)
{
    if (num == 0) return 1
    return (num * factoriel(num - 1))
}

$1 ~ /^Factoriel$/ { print factoriel($2) }
$1 ~ /^Minimum$/        { print minimum ($2, $3) }

function minimum (n,m) {
    return (m < n ? m : n)
}
\end{verbatim}
\end{example}

%%%%%%%%%%%%
\subsection{Les structures de contr{\^o}le}

L'ensemble des \index{awk@\texttt{awk}!structures de contr{\^o}le}structures de contr{\^o}le de "\texttt{awk}" fonctionnent
comme celles du langage C. Nous allons donc ne faire qu'un bref rappel
sur la syntaxe des diff{\'e}rentes structures de contr{\^o}les disponibles.

Dans toute la suite de ce paragraphe, le terme instruction d{\'e}signe un
ensemble d'instructions "\texttt{awk}" s{\'e}par{\'e}es par le caract{\`e}re
"\texttt{;}" ou "\returnkey"et encadr{\'e}es par des "\{","\}".

\index{awk@\texttt{awk}!structures de contr{\^o}le!if@\texttt{if}}
\index{if@\texttt{if}}
\textsl{Structure de contr{\^o}le "\texttt{if}, \texttt{else}"~:}
\begin{quote}
\begin{verbatim}
if (condition)
    instruction
else
    instruction
\end{verbatim}
\end{quote}

\index{awk@\texttt{awk}!structures de contr{\^o}le!while@\texttt{while}}
\index{while@\texttt{while}}
\textsl{Structure de contr{\^o}le "\texttt{while}"~:}
\begin{quote}
\begin{verbatim}
while (condition)
    instruction
\end{verbatim}
\end{quote}

\index{awk@\texttt{awk}!structures de contr{\^o}le!for@\texttt{for}}
\index{for@\texttt{for}}
\textsl{Structure de contr{\^o}le "\texttt{for}"~:}
\begin{quote}
\begin{verbatim}
for (init;condition;it{\'e}ration)
    instruction
\end{verbatim}
\end{quote}

\index{awk@\texttt{awk}!structures de contr{\^o}le!break@\texttt{break}}
\index{break@\texttt{break}}
\noindent \textsl{Instruction "\texttt{break}"~:}
\begin{quote}
L'instruction "\texttt{break}" provoque la sortie du niveau courant d'une
boucle "\texttt{while}" ou "\texttt{for}".
\end{quote}

\index{awk@\texttt{awk}!structures de contr{\^o}le!continue@\texttt{continue}}
\index{continue@\texttt{continue}}
\noindent \textsl{Instruction "\texttt{continue}"~:}
\begin{quote}
L'instruction "\texttt{continue}" provoque l'it{\'e}ration suivante au niveau
courant d'une boucle "\texttt{while}" ou "\texttt{for}".
\end{quote}

\index{awk@\texttt{awk}!structures de contr{\^o}le!next@\texttt{next}}
\index{next@\texttt{next}}
\noindent \textsl{Instruction "\texttt{next}"~:}
L'instruction "\texttt{next}" force "\texttt{awk}" {\`a} passer {\`a} la ligne
suivante du fichier en entr{\'e}e.

\index{awk@\texttt{awk}!exit@\texttt{exit}}
\index{exit@\texttt{exit}}
\noindent \textsl{Instruction "\texttt{exit}"~:}
\begin{quote}
L'instruction "\texttt{exit}" force "\texttt{awk}" \textbf{{\`a} interrompre} la lecture
du fichier d'entr{\'e}e comme si la fin avait {\'e}t{\'e} atteinte.
\end{quote}

\begin{example}
\begin{verbatim}
if ($3 == foo*5) {
    a = $6 % 2;
    print $5, $6, "total", a;
    b = 0;
}
else {
    next
}
while ( i <= NF) {
    print $i;
    i ++;
}
for (i=1; ( i<= NF) && ( i <= 10); i++) {
    if ( i < 0 ) break;
    if ( i == 5) continue;
    print $i;
}
\end{verbatim}
\end{example}

\begin{remarque}
Ces structures de contr{\^o}les et les instructions qui y sont li{\'e}es , bien
que leur nom soit identique {\`a} celui utilis{\'e} pour les structures du shell
sont diff{\'e}rentes en de multiples points~:
\begin{itemize}
	\item	tout d'abord, leur syntaxe est diff{\'e}rente,
	\item	les structures de contr{\^o}les pour "\texttt{awk}" s'ex{\'e}cutent dans
			le processus li{\'e} {\`a} cet utilitaire,
	\item	les structures de contr{\^o}les du shell sont ex{\'e}cut{\'e}es dans le processus
			li{\'e} au script ou bien {\`a} l'interpr{\'e}teur de commandes actif.
\end{itemize}
\textbf{Par cons{\'e}quent, l'utilisation d'un "\texttt{if}" peut {\^e}tre li{\'e} au
shell ou bien {\`a} un utilitaire comme "\texttt{awk}".}
\end{remarque}

%%%%%%%%%%%%%%%%%%%%%%%%%%%%%%%%%%%%%%%%%%%%%%%%%%%%%%%%%%%%
\section{Trucs et astuces}

%%%%%%%%%%%%
\subsection{Commentaires}

Le caract{\`e}re permettant d'introduire des commentaires dans un programme
"\texttt{awk}" est le caract{\`e}re "\verb=#=". Tout ce qui suit ce caract{\`e}re
est ignor{\'e}. La notion de commentaires n'a d'int{\'e}r{\^e}t que lorsque le
programme "\texttt{awk}" se trouve dans un fichier s{\'e}par{\'e}.

%%%%%%%%%%%%
\subsection{Passage de param{\`e}tres du shell vers le programme "\texttt{awk}"}

La notion de passage d'arguments n'a d'int{\'e}r{\^e}t que lorsque le programme
"\texttt{awk}" se trouve dans un fichier s{\'e}par{\'e}. Elle n'est aussi
utilisable que si "\texttt{awk}" re\c{c}oit les donn{\'e}es sur son entr{\'e}e
standard. Dans ce cas, "\texttt{awk}" maintient deux variables~:
\begin{itemize}
	\item	\index{awk@\texttt{awk}!ARGC@\texttt{ARGC}}"\texttt{ARGC}" contenant le nombre d'arguments,
	\item	\index{awk@\texttt{awk}!ARGV@\texttt{ARGV}}"\texttt{ARGV}" tableau contenant la liste des arguments pass{\'e}s au
			programme "\texttt{awk}".
\end{itemize}

\begin{example}
\begin{verbatim}
cat mon.fichier | awk -f program.awk a v=4 b "autre arg"
\end{verbatim}

Dans ce cas, on aura~:\\
\begin{tabular}{|c|c|}
	\hline
		Variable	&
		Valeur		\\
	\hline \hline
		\texttt{ARGC}		&	5					\\
		\verb=ARGV[0]=	&	\texttt{awk}			\\
		\verb=ARGV[1]=	&	\texttt{a}				\\
		\verb=ARGV[2]=	&	\verb,v=4,			\\
		\verb=ARGV[3]=	&	\texttt{b}				\\
		\verb=ARGV[4]=	&	\verb*,autre arg,	\\
	\hline
\end{tabular}
\end{example}

%%%%%%%%%%%%
\subsection{Utilisation de variables du Shell dans le corps du programme
				"\texttt{awk}"}

Cette notion n'est applicable que si "\texttt{awk}" est appel{\'e} {\`a}
l'int{\'e}rieur d'un Shell script et que le corps du programme "{\tt
awk}" ne se trouve pas dans un fichier s{\'e}par{\'e}. Dans ce cas, on aura~:
\begin{quote}
\begin{verbatim}
awk '
    corps du programme awk
' [fichier]
\end{verbatim}
\end{quote}

Pour pouvoir utiliser des variables Shell ou bien le r{\'e}sultat de
commandes, il suffit de les ins{\'e}rer entre deux simples quotes (caract{\`e}re
"\texttt{'}") {\`a} condition que le bloc programme de "awk" soit compris
entre deux simples quotes, ce qui est pr{\'e}f{\'e}rable par rapport aux doubles
quotes (caract{\`e}re """).

\begin{example}
\begin{verbatim}
#!/bin/sh
echo "Entrez une chaine :\c"
read answer
echo $answer | awk '
    BEGIN {
        status=0;
        today='`date +%H%M`'
    }
    NR > 1 { status=1}
    $1 ~ /[0-9].*/ {
        printf ("La chaine %s comprend une partie num{\'e}rique: %d\n",
            '$answer', $1);
    }
    END {
        printf ("Execut{\'e}e le %s\n", today)
    }
'
\end{verbatim}
\end{example}

\noindent{\Large \textbf{ATTENTION}}


%%%%%%%%%%%%%%%%%%%%%%%%%%%%%%%%%%%%%%%%%%%%%%%%%%%%%%%%%%%%
\section{\label{adv-fltrs-awk-ex}Exemple}

Soit le fichier "\texttt{donjon}"~:
\begin{quote}
\begin{tabular}{|l|}
	\hline
	\verb=LANCELOT:DULAC:chevalier:200=	\\
	\verb=GODEFROY:DE BOUILLON:chevalier:300=	\\
	\verb=ARTHUR::roi:150=	\\
	\verb=MERLIN:L'ENCHANTEUR:magicien:0=	\\
	\verb=GODEFROY:DE BOUILLON:chevalier:400=	\\
	\verb=LANCELOT:DULAC:chevalier:300=	\\
	\hline
\end{tabular}
\end{quote}

Soit le fichier programme "\texttt{cumul.awk}"~:
\begin{quote}
\begin{verbatim}
BEGIN {
    FS=":"
    print "Statistiques sur le nombre de sarrasins occis"
    nb=0
}

$1 ~ /[a-zA-Z]/ {
    cumul[$1] += $4
    for (i=1; (i <= nb) && (list[i] != $1; i++)
    if ( i > nb) {
        nb ++
        list[nb]=$1 $2
    }
}
END {
    printf ("%.20s\t%.20s \n","Zigoto", "Cumul")
    for (i=1; i <= nb; i++)
        printf ("%.20s\t%.20s \n",
             list[i], cumul[list[i]])
}
\end{verbatim}
\end{quote}

La commande
\begin{center}
\begin{verbatim}
awk -f cumul.awk donjon
\end{verbatim}
\end{center}
produit le r{\'e}sultat suivant~:
\begin{quote}
\begin{verbatim}
Statistiques sur le nombre de sarrasins occis
Zigoto                  Cumul
LANCELOT DULAC          500
GODEFROY DE BOUILLON    700
ARTHUR                  150
MERLIN L'ENCHANTEUR     0
\end{verbatim}
\end{quote}

%%%%%%%%%%%%%%%%%%%%%%%%%%%%%%%%%%%%%%%%%%%%%%%%%%%%%%%%%%%%
\section{\label{adv-fltrs-awk-f}Remarque sur l'utilisation de l'option "\texttt{-f}"}

{\`A} la r{\`e}gle 10 explicit{\'e}e {\`a} la section \ref{pgm-intro-basic-rules}, nous
avions pr{\'e}ciser que la premi{\`e}re ligne d'un script devait sp{\'e}cifier le nom
de l'ex{\'e}cutable associ{\'e} au shell utiliser.

Nous avons vu de m{\^e}me, que la commande "\texttt{awk}" dispose de l'option
"\texttt{-f}", permettant de pr{\'e}ciser un fichier de requ{\^e}tes "\texttt{awk}".

Si l'analogie est faite avec le shell, le processus charg{\'e} d'{\'e}valuer chaque
ligne du fichier est l'ex{\'e}cutable sp{\'e}cifi{\'e} au niveau de la premi{\`e}re
ligne du fichier contenant les instructions. Par cons{\'e}quent si le fichier
contenant les requ{\^e}tes "\texttt{awk}"~:
\begin{itemize}
	\item	a comme premi{\`e}re ligne "\verb=#!/usr/bin/awk="\footnote{Nous
			consid{\'e}rons ici que l'ex{\'e}cutable de "\texttt{awk}" se trouve dans
			le r{\'e}pertoire "\texttt{/usr/bin}". Pour en conna{\^\i}tre la
			localisation exacte sur votre syst{\`e}me, vous pouvez utiliser
			la commande "\texttt{which(1)}" ou "\texttt{whereis(1)}".},
	\item	est accessible en ex{\'e}cution,
\end{itemize}
il peut {\^e}tre consid{\'e}r{\'e} comme un filtre ex{\'e}cutant les requ{\^e}tes "\texttt{awk}"
qu'il contient.

Par exemple, si cette notion est appliqu{\'e}e au fichier d{\'e}crit dans l'exemple
de la section \ref{adv-fltrs-awk-ex}, le contenu du fichier
"\texttt{cumul.awk}"~:
\begin{quote}
\begin{verbatim}
#!/usr/bin/awk
BEGIN {
    FS=":"
    print "Statistiques sur le nombre de sarrasins occis"
    nb=0
}

$1 ~ /[a-zA-Z]/ {
    cumul[$1] += $4
    for (i=1; (i <= nb) && (list[i] != $1; i++)
    if ( i > nb) {
        nb ++
        list[nb]=$1 $2
    }
}
END {
    printf ("%.20s\t%.20s \n","Zigoto", "Cumul")
    for (i=1; i <= nb; i++)
        printf ("%.20s\t%.20s \n",
             list[i], cumul[list[i]])
}
\end{verbatim}
\end{quote}

La commande qui sera saisie deviendra~:
\begin{quote}
\begin{verbatim}
cumul.awk donjon
\end{verbatim}
\end{quote}

Si nous voulons utiliser le programme "\texttt{cumul.awk}" comme un filtre,
il sera alors possible de saisir la commande suivante~:
\begin{quote}
\begin{verbatim}
cat donjon | cumul.awk
\end{verbatim}
\end{quote}

Dans tous les cas, le r{\'e}sultat obtenu sur la sortie standard sera identique {\`a}
celui montr{\'e} {\`a} la section \ref{adv-fltrs-awk-ex}.
