%%%%%%%%%%%%%%%%%%%%%%%%%%%%%%%%%%%%%%%%%%%%%%%%%%%%%%%%%%%%%%%%%%%%%
%
% Annexes
%
\clearpage
\newpage
\begin{appendix}

%%%%%%%%%%%%%%%%%%%%%%%%%%%%%%%%%%%%%%%%%%%%%%%%%%%%%%%%%%%%%%%%%%%%%
\chapter{\label{ann-def-structures}Prototype des structures
utilis{\'e}es pour les
{\sl BSD Sockets}}

%%%%%%%%%%%%%%%%%%%%%%%%%%%%%%%%%%%%%%%%%%%%%%%%%%%%%%%%%%%%%%%%%%%%%
\section{\label{ann-struct-sockaddrin}Structure {\tt sockaddr\_in}}

La structure {\tt sockaddr\_in} se compose des champs
suivants:
\begin{quote}
\begin{tabular}{|l|l|p{8cm}|}
	\hline
	Type			&	Nom du champ	&	Description	\\
	\hline
	{\tt short int}		&	{\tt sin\_family}	& 
		Sp{\'e}cifie la famille d'adresses. Dans le cas du protocole IP, on 
		initialise ce champ {\`a} la valeur de la constante {\tt AF\_INET}.\\
	{\tt unsigned short}	&	{\tt sin\_port}		& 
		Contient le num{\'e}ro du port de communication {\`a} utiliser.\\
	{\tt struct in\_addr}	&	{\tt sin\_addr}		& 
		Structure permettant de pr{\'e}ciser l'adresse IP (cf. section
		\ref{ann-struct-inaddr}) \\
	{\tt char}				&	{\tt sin\_zero[8]}	&
		Zone de {\it padding} non utilis{\'e}e dans le cas de IP. Sa
		pr{\'e}sence dans la structure <<~{\tt sockadd\_in}~>> permet {\`a} toutes
		les structures ``{\tt sockaddr\_{\it XX}}'' d'avoir la m{\^e}me taille.\\
	\hline
\end{tabular}
\end{quote}

\begin{remarque}
Pour chaque type de famille d'adresses, on aura une structure ``{\tt
sockaddr}''. L'identifiant {\`a} la fin du nom du prototype de la structure
indique la famille d'adresses concern{\'e}e. Par exemple~:

\begin{tabular}{lp{8cm}}
{\tt in}	&	Structure associ{\'e}e {\`a} IP	\\
{\tt un}	&	Structure associ{\'e}e {\`a} la famille {\tt AF\_UNIX} permettant de faire
		de la communication entre processus sur une m{\^e}me machine (cf. chapitre
		\ref{unstream})
\end{tabular}
\end{remarque}

%%%%%%%%%%%%%%%%%%%%%%%%%%%%%%%%%%%%%%%%%%%%%%%%%%%%%%%%%%%%%%%%%%%%%
\section{\label{ann-struct-inaddr}Structure {\tt in\_addr}}

La structure {\tt in\_addr} a le format suivant~:
\begin{quote}
\begin{verbatim}
struct in_addr {
    union {
        struct { u_char s_b1,s_b2,s_b3,s_b4; } S_un_b;
        struct { u_short s_w1,s_w2; } S_un_w;
        u_long S_addr;
    } S_un;
#define s_addr  S_un.S_addr         /* can be used for most tcp & ip
code */
#define s_host  S_un.S_un_b.s_b2    /* host on imp */
#define s_net   S_un.S_un_b.s_b1    /* network */
#define s_imp   S_un.S_un_w.s_w2    /* imp */
#define s_impno S_un.S_un_b.s_b4    /* imp # */
#define s_lh    S_un.S_un_b.s_b3    /* logical host */
};
\end{verbatim}
\end{quote}
o{\`u} 
\begin{itemize}
	\item {\tt u\_long} d{\'e}signe un type entier non sign{\'e} sur 32 bits,
	\item {\tt u\_word} d{\'e}signe un type entier non sign{\'e} sur 16 bits,
	\item {\tt u\_char} d{\'e}signe un type entier non sign{\'e} sur 8 bits,
\end{itemize}

Toutefois, sur la majeure partie des syst{\`e}mes {\Unix} actuels, cette structure
ne comporte plus qu'un seul et unique champ. Elle se r{\'e}duit {\`a}~:
\begin{quote}
\begin{verbatim}
struct in_addr {
    u_long s_addr;
};
\end{verbatim}
\end{quote}

Elle reste donc compatible avec le format d{\'e}crit pr{\'e}c{\'e}demment gr{\^a}ce aux diff{\'e}rentes
{\it macros}.

Toutefois, la quasi-totalit{\'e} des appels syst{\`e}mes manipulant des adresses IP, renverrons une
adresse stock{\'e}e sur un entier non-sign{\'e} de 32 bits.
Cela correspond bien {\`a} l'unique champ qui reste dans cette <<{\sl structure}>>.

%%%%%%%%%%%%%%%%%%%%%%%%%%%%%%%%%%%%%%%%%%%%%%%%%%%%%%%%%%%%%%%%%%%%%
\section{\label{ann-struct-sockaddrun}Structure {\tt sockaddr\_un}}

La structure {\tt sockaddr\_un} a le format suivant~:
\begin{quote}
\begin{verbatim}
struct sockaddr_un {
    unsigned short sun_family;      /* AF_UNIX */
    char sun_path[UNIX_PATH_MAX];   /* pathname */
};
\end{verbatim}
\end{quote}

Le champ {\tt sun\_family} permet de sp{\'e}cifier la famille d'adresse. Le
champ {\tt sun\_path} contient le chemin d'acc{\`e}s au fichier servant de {\sl port de communication}
entre les processus locaux.

\begin{remarque}
La constante {\tt UNIX\_PATH\_MAX} d{\'e}pend des syst{\`e}mes. Par exemple, sous
Linux 2.0, ce param{\`e}tre a pour valeur 105.
\end{remarque}

%%%%%%%%%%%%%%%%%%%%%%%%%%%%%%%%%%%%%%%%%%%%%%%%%%%%%%%%%%%%%%%%%%%%%
\chapter{\label{ann-part-manunix}Pages particuli{\`e}res du manuel {\sc Unix}}

Les pages qui vont suivre sont extraites du manuel {\Unix} livr{\'e}es
avec {\sl Linux 2.0}. Elles sont donc en anglais.

%%%%%%%%%%%%%%%
\section{Pages du manuel de la section 2 (appels syst{\`e}mes)}

%%%%%%%%%%%%%%%
\subsection{Manuel de {\tt accept()}}
\begin{english}
{\sl ACCEPT(2)}\\

{\sl NAME}
\begin{quote}
 {\tt accept} - accept a connection on a socket
\end{quote}

{\sl SYNOPSIS}
\begin{quote}
\begin{verbatim}
#include <sys/types.h>
#include <sys/socket.h>
int accept(int s, struct sockaddr *addr, int *addrlen);
\end{verbatim}
\end{quote}

{\sl DESCRIPTION}
\begin{quote}
	The argument {\tt s} is a socket that has been created with {\tt
socket(2)},
	bound to an address with {\tt bind(2)}, and is listening for 
	connections after a {\tt listen(2)}. The accept argument extracts the 
	first connection request on the queue of pending connections,
	creates a new socket with the same properties of s and allocates a
	new file descriptor for the socket. If no pending connections are
	present on the queue, and the socket is not marked as 
	non-blocking, {\tt accept} blocks the caller until a connection is
	present. If the socket is marked non-blocking and no pending
	connections are present on the queue, accept returns an error as
	described below. The accepted socket may not be used to accept 
	more connections. The original socket s remains open.

	The argument {\tt addr} is a result parameter that is filled in with the
	address of the connecting entity, as known to the communications
	layer. The exact format of the addr parameter is determined by
	the domain in which the communication is occurring. The {\tt addrlen} is
	a value-result parameter; it should initially contain the 
	amount of space pointed to by {\tt addr}; on return it will contain the 
	actual length (in bytes) of the address returned. This call is
	used with connection-based socket types, currently with
	{\tt SOCK\_STREAM}.

	It is possible to {\tt select(2)} a socket for the purposes of doing an
	{\tt accept} by selecting it for read.

	For certain protocols which require an explicit confirmation, such
	as ISO or DATAKIT, accept can be thought of as merely dequeuing the
	 next connection request and not implying confirmation. 
	Confirmation can be implied by a normal read or write on the new 
	file descriptor, and rejection can be implied by closing the new
	socket.

	One can obtain user connection request data without confirming the
	connection by issuing a {\tt recvmsg(2)} call with an {\tt msg\_iovlen} of 0
	and a non-zero {\tt msg\_controllen}, or by issuing a {\tt getsockopt(2)}
	request. Similarly, one can provide user connection rejection
	information by issuing a {\tt sendmsg(2)} call with providing only the 
	control information, or by calling {\tt setsockopt(2)}.
\end{quote}

{\sl RETURN VALUES}
\begin{quote}
	The call returns -1 on error. If it succeeds, it returns a
	non-negative integer that is a descriptor for the accepted
	socket.
\end{quote}

{\sl ERRORS}
\begin{quote}
\begin{tabular}{lp{8cm}}
	{\tt EBADF}		&	The descriptor is invalid.\\
	{\tt ENOTSOCK}	&	The descriptor references a file, not a socket.\\
	{\tt EOPNOTSUPP}	&	The referenced socket is not of type {\tt SOCK\_STREAM}.\\
	{\tt EFAULT}		&	The addr parameter is not in a writable part of
							the user address space.\\
	{\tt EWOULDBLOCK}	&	The socket is marked non-blocking and no connections
							are present to be accepted.
\end{tabular}
\end{quote}

{\sl HISTORY}
\begin{quote}
 The accept function appeared in BSD 4.2.
\end{quote}

{\sl SEE ALSO}
\begin{quote}
	{\tt bind(2)}, {\tt connect(2)}, {\tt listen(2)}, {\tt select(2)},
	{\tt socket(2)}
\end{quote}
\end{english}

%%%%%%%%%%%%%%%
\subsection{Manuel de {\tt bind()}}
\begin{english}
{\sl BIND(2)}\\

{\sl NAME}
\begin{quote}
	{\tt bind} - bind a name to a socket
\end{quote}

{\sl SYNOPSIS}
\begin{quote}
\begin{verbatim}
#include <sys/types.h>
#include <sys/socket.h>

int bind(int sockfd, struct sockaddr *my_addr, int addrlen);
\end{verbatim}
\end{quote}

{\sl DESCRIPTION}
\begin{quote}
	{\tt bind} gives the socket, sockfd, the local address {\tt my\_addr}.
	{\tt my\_addr} is addrlen bytes long. Traditionally, this is
	called ``assigning a name to a socket'' (when a socket is
	created with {\tt socket(2)}, it exists in a name space (address
	family) but has no name assigned.)
\end{quote}

{\sl NOTES}
\begin{quote}
	Binding a name in the {\Unix} domain creates a socket in the
	file system that must be deleted by the caller when it is
	no longer needed (using {\tt unlink(2)}).

	The rules used in name binding vary between communication
	domains. Consult the manual entries in section 4 for
	detailed information.
\end{quote}

{\sl RETURN VALUE}
\begin{quote}
	On success, zero is returned. On error, -1 is returned,
	and {\tt errno} is set appropriately.
\end{quote}

{\sl ERRORS}
\begin{quote}
\begin{tabular}{lp{8cm}}
	{\tt EBADF}		& {\tt sockfd} is not a valid descriptor.\\
	{\tt EINVAL}	& The socket is already bound to an address. This
				may change in the future: see {\tt linux/unix/sock.c}
				for details.\\
	{\tt EACCES}	& The address is protected, and the user is not the
				super-user.
\end{tabular}

	The following errors are specific to {\Unix} domain ({\tt AF\_UNIX})
	sockets:\\
\begin{tabular}{lp{8cm}}
	{\tt EINVAL}	& The {\tt addr\_len} was wrong, or the socket was not in
			the {\tt AF\_UNIX} family.\\
	{\tt EROFS}		& The socket inode would reside on a read-only file
			system.\\
	{\tt EFAULT}	& {\tt my\_addr} points outside your accessible address
			space.\\
	{\tt ENAMETOOLONG}	& {\tt my\_addr} is too long.\\
	{\tt ENOENT}	& The file does not exist.\\
	{\tt ENOMEM}	& Insufficient kernel memory was available.\\
	{\tt ENOTDIR}	& A component of the path prefix is not a directory.\\
	{\tt EACCES}	& Search permission is denied on a component of the
			path prefix.\\
	{\tt ELOOP}		& {\tt my\_addr} contains a circular reference (i.e., via a
			symbolic link) \\
\end{tabular}
\end{quote}

{\sl HISTORY}
\begin{quote}
	The {\tt bind} function call appeared in BSD 4.2.
\end{quote}

{\sl SEE ALSO}
\begin{quote}
	{\tt accept(2)}, {\tt connect(2)}, {\tt listen(2)}, {\tt socket(2)}, 
	{\tt getsockname(2)}
\end{quote}
\end{english}

%%%%%%%%%%%%%%%
\subsection{Manuel de {\tt close}}
\begin{english}
{\sl CLOSE(2)}\\

{\sl NAME}
\begin{quote}
	{\tt close} - close a file descriptor
\end{quote}

{\sl SYNOPSIS}
\begin{quote}
\begin{verbatim}
#include <unistd.h>

int close(int fd);
\end{verbatim}
\end{quote}

{\sl DESCRIPTION}
\begin{quote}
	{\tt close} closes a file descriptor, so that it no longer
	refers to any file and may be reused.

	If {\tt fd} is the last copy of a particular file descriptor the
	resources associated with it are freed; for example any
	locks held are removed and if the descriptor was the last
	reference to a file which has been removed using unlink
	the file is deleted.
\end{quote}

{\sl RETURN VALUE}
\begin{quote}
	{\tt close} returns zero on success, or -1 if an {\tt error} occurred.
\end{quote}

{\sl ERRORS}
\begin{quote}
\begin{tabular}{lp{8cm}}
	{\tt EBADF}	& {\tt fd} isn't a valid open file descriptor.
\end{tabular}
\end{quote}

{\sl CONFORMING TO}
\begin{quote}
	SVID, AT\&T, POSIX, X/OPEN, BSD 4.3
\end{quote}

{\sl NOTES}
\begin{quote}
	Not checking the return value of {\tt close} is a common but
	nevertheless serious programming error. File system
	implementations which use techniques as ``write-behind''
	to increase performance may lead to {\tt write(2)} succeeding,
	although the data has not been written yet. The error
	status may be reported at a later write operation, but it
	is guaranteed to be reported on closing the file. Not
	checking the return value when closing the file may lead
	to silent loss of data. This can especially be observed
	with NFS and disk quotas.
\end{quote}

{\sl SEE ALSO}
\begin{quote}
	{\tt open(2)}, {\tt fcntl(2)}, {\tt shutdown(2)}, {\tt unlink(2)},
	{\tt fclose(3)}.
\end{quote}
\end{english}

%%%%%%%%%%%%%%%
\subsection{Manuel de {\tt connect()}}
\begin{english}
{\sl CONNECT(2)}\\

{\sl NAME}
\begin{quote}
	{\tt connect} - initiate a connection on a socket
\end{quote}

{\sl SYNOPSIS}
\begin{quote}
\begin{verbatim}
#include <sys/types.h>
#include <sys/socket.h>

int connect(int sockfd, struct sockaddr *serv_addr,
            int addrlen );
\end{verbatim}
\end{quote}

{\sl DESCRIPTION}
\begin{quote}
	The parameter {\tt sockfd} is a socket. If it is of type {\tt
	SOCK\_DGRAM}, this call specifies the peer with which the socket
	is to be associated; this address is that to which datagrams are
	to be sent, and the only address from which datagrams are to be
	received. If the socket is of type {\tt SOCK\_STREAM}, this
	call attempts to make a connection to another socket. The other 
	socket is specified by {\tt serv\_addr}, which is an address in
	the communications space of the socket. Each communications space 
	interprets the {\tt serv\_addr}, parameter in its own way. 
	Generally, stream sockets may successfully connect only once;
	datagram sockets may use connect multiple times to change their
	association. Datagram sockets may dissolve the association by
	connecting to an invalid address, such as a null address.
\end{quote}

{\sl RETURN VALUE}
\begin{quote}
	If the connection or binding succeeds, zero is returned.
	On error, -1 is returned, and {\tt errno} is set appropriately.
\end{quote}

{\sl ERRORS}
\begin{quote}
	See the Linux kernel source code for details.
\end{quote}

{\sl HISTORY}
\begin{quote}
	The connect function call first appeared in BSD 4.2.
\end{quote}

{\sl SEE ALSO}
\begin{quote}
	{\tt accept(2)}, {\tt bind(2)}, {\tt listen(2)}, {\tt socket(2)},
	{\tt getsockname(2)}
\end{quote}
\end{english}

%%%%%%%%%%%%%%%
\subsection{Manuel de {\tt ioctl()}}
\begin{english}
{\sl IOCTL(2)}\\

{\sl NAME}
\begin{quote}
	{\tt ioctl} - control device
\end{quote}

{\sl SYNOPSIS}
\begin{quote}
\begin{verbatim}
#include <sys/ioctl.h>

int ioctl(int d, int request, ...)
\end{verbatim}

The ``third'' argument is traditionally {\tt char *argp}, and
will be so named for this discussion.
\end{quote}

{\sl DESCRIPTION}
\begin{quote}
	The {\tt ioctl} function manipulates the underlying device
	parameters of special files. In particular, many operating
	characteristics of character special files (e.g. terminals) may be
	controlled with {\tt ioctl} requests. The argument {\tt d} must be
	an open file descriptor.

	An {\tt ioctl} request has encoded in it whether the argument is an 
	in parameter or out parameter, and the size of the argument {\tt
	argp} in bytes. Macros and defines used in specifying an {\tt
	ioctl} request are located in the file {\tt sys/ioctl.h}.
\end{quote}

{\sl RETURN VALUE}
\begin{quote}
	On success, zero is returned. On error, -1 is returned,
	and {\tt errno} is set appropriately.
\end{quote}

{\sl ERRORS}
\begin{quote}
\begin{tabular}{lp{8cm}}
	{\tt EBADF}		& {\tt d} is not a valid descriptor.\\
	{\tt ENOTTY}	& {\tt d} is not associated with a character special
					  device.\\
	{\tt ENOTTY}	& The specified request does not apply to the kind of
					  object that the descriptor d references.\\
	{\tt EINVAL}	& Request or {\tt argp} is not valid.
\end{tabular}
\end{quote}

{\sl HISTORY}
\begin{quote}
	An {\tt ioctl} function call appeared in Version 7 AT\&T {\Unix}.
\end{quote}

{\sl SEE ALSO}
\begin{quote}
	{\tt execve(2)}, {\tt fcntl(2)}, {\tt mt(4)}, {\tt sd(4)}, {\tt tty(4)}
\end{quote}
\end{english}

%%%%%%%%%%%%%%%
\subsection{Manuel de {\tt kill()}}
\begin{english}
{\sl KILL(2)}\\

{\sl NAME}
\begin{quote}
	{\tt kill} - send signal to a process
\end{quote}

{\sl SYNOPSIS}
\begin{quote}
\begin{verbatim}
#include <signal.h>

int kill(pid_t pid, int sig);
\end{verbatim}
\end{quote}

{\sl DESCRIPTION}
\begin{quote}
	{\tt kill()} can be used to send any signal to any process group or
	process.

	If {\tt pid} is positive, then signal {\tt sig} is sent to {\tt
	pid}. In this case, $0$ is returned on success, or a negative
	value on error.

	If {\tt pid} equals $-1$, then {\tt sig} is sent to every process
	except for the first one, from higher numbers in the proc table to
	lower. In this case, $0$ is returned on success, or the last error
	condition from {\tt send\_sig()} is returned.

	If {\tt pid} is less than $-1$, then {\tt sig} is sent to every
	process in the process group {\tt -pid}. In this case, the number
	 of processes the signal was sent to is returned, or a negative
	value for failure.
\end{quote}

{\sl RETURN VALUE}
\begin{quote}
	On success, zero is returned. On error, $-1$ is returned, and {\tt
	errno} is set appropriately.
\end{quote}

{\sl ERRORS}
\begin{quote}
\begin{tabular}{lp{8cm}}
	{\tt EINVAL}	&	An invalid signal is sent.\\
	{\tt ESRCH}		&	The {\tt pid} or process group does not exist.\\
	{\tt EPERM}		&	The effective userid of the process calling
						{\tt kill()} is not equal to the effective userid of
						{\tt pid}, unless the superuser called {\tt kill()}.
\end{tabular}
\end{quote}

{\sl BUGS}
\begin{quote}
	It is impossible to send a signal to task number one, the {\tt
	init} process, for which it has not installed a signal handler. 
	This is done to assure the system is not brought down
	accidentally.
\end{quote}

{\sl CONFORMING TO}
\begin{quote}
	SVID, AT\&T, POSIX.1, X/OPEN, BSD 4.3
\end{quote}

{\sl SEE ALSO}
\begin{quote}
	{\tt \_exit(2)}, {\tt exit(2)}, {\tt signal(2)}, {\tt signal(7)}
\end{quote}
\end{english}

%%%%%%%%%%%%%%%
\subsection{Manuel de {\tt listen()}}
\begin{english}
{\sl LISTEN(2)}\\

{\sl NAME}
\begin{quote}
	{\tt listen} - listen for connections on a socket
\end{quote}

{\sl SYNOPSIS}
\begin{quote}
\begin{verbatim}
#include <sys/socket.h>

int listen(int s, int backlog);
\end{verbatim}
\end{quote}

{\sl DESCRIPTION}
\begin{quote}
	To accept connections, a socket is first created with {\tt
	socket(2)}, a willingness to accept incoming connections and a 
	queue limit for incoming connections are specified with listen, and
	then the connections are accepted with {\tt accept(2)}. The 
	listen call applies only to sockets of type {\tt SOCK\_STREAM} or
	{\tt SOCK\_SEQPACKET}.

	The {\tt backlog} parameter defines the maximum length the queue of 
	pending connections may grow to. If a connection request
	arrives with the queue full the client may receive an error with 
	an indication of {\tt ECONNREFUSED}, or, if the underlying protocol
	supports retransmission, the request may be ignored so that
	retries may succeed.
\end{quote}

{\sl RETURN VALUE}
\begin{quote}
	On success, zero is returned. On error, -1 is returned,
	and {\tt errno} is set appropriately.
\end{quote}

{\sl ERRORS}
\begin{quote}
\begin{tabular}{lp{8cm}}
	{\tt EBADF}			& The argument {\tt s} is not a valid descriptor.\\
	{\tt ENOTSOCK}		& The argument {\tt s} is not a socket.\\
	{\tt EOPNOTSUPP}	& The socket is not of a type that supports the
						  operation {\tt listen}.
\end{tabular}
\end{quote}

{\sl HISTORY}
\begin{quote}
	The {\tt listen} function call appeared in BSD 4.2.
\end{quote}

{\sl BUGS}
\begin{quote}
	If the socket is of type {\tt af\_inet}, and the {\tt backlog}
	argument is greater than 128 it is silently truncated to  128.  For
	portable  applications  don't rely on this value since BSD (and at
	least some BSD derived systems) limit the  {\tt backlog} to 5.
\end{quote}

{\sl SEE ALSO}
\begin{quote}
	{\tt accept(2)}, {\tt connect(2)}, {\tt socket(2)}
\end{quote}
\end{english}

%%%%%%%%%%%%%%%
\subsection{Manuel de {\tt open()}}
\begin{english}
{\sl OPEN(2)}\\

{\sl NAME}
\begin{quote}
	{\tt open}, {\tt creat} - open and possibly create a file or device
\end{quote}

{\sl SYNOPSIS}
\begin{quote}
\begin{verbatim}
#include <sys/types.h>
#include <sys/stat.h>
#include <fcntl.h>

int open(const char *pathname, int flags);
int open(const char *pathname, int flags, mode_t mode);
int creat(const char *pathname, mode_t mode);
\end{verbatim}
\end{quote}

{\sl DESCRIPTION}
\begin{quote}
	{\tt open} attempts to open a file and return a file descriptor (a
	small, non-negative integer for use in {\tt read}, {\tt write},
	etc.)

	{\tt flags} is one of {\tt O\_RDONLY}, {\tt O\_WRONLY} or {\tt
	O\_RDWR} which request opening the file read-only, write-only or
	read/write, respectively.

	{\tt flags} may also be bitwise-or'd with one or more of the
	following:
	\begin{description}
		\item[{\tt O\_CREAT}]\mbox{}\\
		If the file does not exist it will be created. \\

		\item[{\tt O\_EXCL}]\mbox{}\\
		When used with {\tt O\_CREAT}, if the file already exists
		it is an error and the open will fail. See {\sl BUGS}
		below, though. \\

		\item[{\tt O\_NOCTTY}]\mbox{}\\
		If pathname refers to a terminal device -- see
		{\tt tty(4)} -- it will not become the process's controlling
		terminal even if the process does not have one.\\

		\item[{\tt O\_TRUNC}]\mbox{}\\
		If the file already exists it will be truncated. \\

		\item[{\tt O\_APPEND}]\mbox{}\\
		The file is opened in append mode. Initially, and
		before each write, the file pointer is positioned
		at the end of the file, as if with {\tt lseek}. \\

		\item[{\tt O\_NONBLOCK} or {\tt O\_NDELAY}]\mbox{}\\
		The file is opened in non-blocking mode. Neither
		the open nor any subsequent operations on the file
		descriptor which is returned will cause the calling
		process to wait. \\

		\item[{\tt O\_SYNC}]\mbox{}\\
		The file is opened for synchronous I/O. Any writes
		on the resulting file descriptor will block the
		calling process until the data has been physically
		written to the underlying hardware. See {\sl BUGS}
		below, though.
	\end{description}

	Some of these optional flags can be altered using {\tt fcntl}
	after the file has been opened.

	{\tt mode} specifies the permissions to use if a new file is
	created. It is modified by the process's umask in the usual
	way: the permissions of the created file are (mode \& \~umask).

	{\tt mode} should always be specified when {\tt O\_CREAT} is in 
	the flags, and is ignored otherwise.

	{\tt creat} is equivalent to open with flags equal to {\tt
	O\_CREAT}\verb=|={\tt O\_WRONLY}\verb=|={\tt O\_TRUNC}.
\end{quote}

{\sl RETURN VALUE}
\begin{quote}
	{\tt open} and {\tt creat} return the new file descriptor, or -1 if
	an error occurred (in which case, {\tt errno} is set appropriately).
\end{quote}

{\sl ERRORS}
\begin{quote}
\begin{description}
	\item[{\tt EEXIST}]\mbox{}\\
		{\tt pathname} already exists and {\tt O\_CREAT} and {\tt O\_EXCL}
were
		used.

	\item[{\tt EISDIR}]\mbox{}\\
		{\tt pathname} refers to a directory and the access
		requested involved writing.

	\item[{\tt ETXTBSY}]\mbox{}\\
		{\tt pathname} refers to an executable image which is
		currently being executed and write access was requested.

	\item[{\tt EFAULT}]\mbox{}\\
		{\tt pathname} points outside your accessible address space.

	\item[{\tt EACCES}]\mbox{}\\
		The requested access to the file is not allowed, or one of the
		directories in pathname did not allow search ({\sl execute})
permission.

	\item[{\tt ENAMETOOLONG}]\mbox{}\\
		{\tt pathname} was too long.

	\item[{\tt ENOENT}]\mbox{}\\
		A directory component in pathname does not exist or is a dangling
		symbolic link.

	\item[{\tt ENOTDIR}]\mbox{}\\
		A component used as a directory in {\tt pathname} is not, in fact,
		a directory.

	\item[{\tt EMFILE}]\mbox{}\\
		The process already has the maximum number of files open.

	\item[{\tt ENFILE}]\mbox{}\\
		The limit on the total number of files open on the system has
		been reached.

	\item[{\tt ENOMEM}]\mbox{}\\
		Insufficient kernel memory was available.

	\item[{\tt EROFS}]\mbox{}\\
		{\tt pathname} refers to a file on a read-only filesystem and
		write access was requested.

	\item[{\tt ELOOP}]\mbox{}\\
		{\tt pathname} contains a reference to a circular symbolic
		link, ie a symbolic link whose expansion contains a reference to itself.

	\item[{\tt ENOSPC}]\mbox{}\\
		{\tt pathname} was to be created but the device containing pathname
		has no room for the new file.
\end{description}
\end{quote}

{\sl CONFORMING TO}
\begin{quote}
	SVID, AT\&T, POSIX, X/OPEN, BSD 4.3
\end{quote}

{\sl BUGS}
\begin{quote}
	There are many infelicities in the protocol underlying NFS, 
	affecting amongst others {\tt O\_SYNC}, {\tt O\_NDELAY}, and
	{\tt O\_APPEND}.

	{\tt O\_EXCL} is broken on NFS file systems, programs which rely on
	 it for performing locking tasks will contain a race condition. 
	The solution for performing atomic file locking using a {\sl lockfile}
	is to create a unique file on the same fs (e.g., incorporating 
	{\sl hostname} and {\sl pid}), use {\tt link(2)} to make a link to the
	lockfile and use {\tt stat(2)} on the unique file to check if its
	link count has increased to 2. Do not use the return value of the
	{\tt link()} call.
\end{quote}

{\sl SEE ALSO}
\begin{quote}
	{\tt read(2)}, {\tt write(2)}, {\tt fcntl(2)}, {\tt close(2)}, {\tt
	unlink(2)}, {\tt mknod(2)}, {\tt stat(2)}, {\tt umask(2)}, {\tt
	mount(2)}, {\tt socket(2)}, {\tt socket(2)}, {\tt fopen(3)}, {\tt
	link(2)}.
\end{quote}
\end{english}

%%%%%%%%%%%%%%%
\subsection{Manuel de {\tt read()}}
\begin{english}
{\sl READ(2)}\\

{\sl NAME}
\begin{quote}
	{\tt read} - read from a file descriptor
\end{quote}

{\sl SYNOPSIS}
\begin{quote}
\begin{verbatim}
#include <sys/types.h>
#include <unistd.h>

int read(int fd, char *buf, size_t count);
\end{verbatim}
\end{quote}

{\sl DESCRIPTION}
\begin{quote}
	{\tt read} reads up to count bytes from file descriptor {\tt
	fd} into the buffer starting at {\tt buf}.
\end{quote}

{\sl RETURN VALUE}
\begin{quote}
	On success, the number of bytes read are returned (zero indicates
	end of file). On error, -1 is returned, and {\tt errno} is set
	appropriately.
\end{quote}

{\sl ERRORS}
\begin{quote}
\begin{tabular}{lp{8cm}}
	{\tt EINTR}		&	The call was interrupted by a signal before any
						data was read.\\
	{\tt EAGAIN}	&	Non-blocking I/O has been selected using
						{\tt O\_NONBLOCK} and no data was immediately
						available for reading.\\
	{\tt EISDIR}	&	{\tt fd} refers to a directory.\\
	{\tt EBADF}		&	{\tt fd} is not a valid file descriptor or is not open
						for reading.\\
	{\tt EINVAL}	&	{\tt fd} is attached to an object which is unsuitable
						for reading.\\
	{\tt EFAULT}	&	{\tt buf} is outside your accessible address space.\\
\end{tabular}
	Other errors may occur, depending on the object connected to {\tt
	fd}.
\end{quote}

{\sl CONFORMING TO}
\begin{quote}
	SVID, AT\&T, POSIX, X/OPEN, BSD 4.3
\end{quote}

{\sl SEE ALSO}
\begin{quote}
	{\tt readdir(2)}, {\tt write(2)}, {\tt write(2)}, {\tt fcntl(2)},
	{\tt close(2)}, {\tt lseek(2)}, {\tt select(2)}, {\tt readlink(2)},
	{\tt ioctl(2)}, {\tt fread(3)}.
\end{quote}
\end{english}

%%%%%%%%%%%%%%%
\subsection{Manuel de {\tt recv()} et de {\tt recvfrom()}}
\begin{english}
{\sl RECV(2)}\\

{\sl NAME}
\begin{quote}
 	{\tt recv}, {\tt recvfrom}, {\tt recvmsg} - receive a message from a
socket
\end{quote}

{\sl SYNOPSIS}
\begin{quote}
\begin{verbatim}
#include <sys/types.h>
#include <sys/socket.h>

int recv(int s, void *buf, int len, unsigned int flags);

int recvfrom(int s, void *buf, int len, unsigned int flags
             struct sockaddr *from, int *fromlen);

int recvmsg(int s, struct msghdr *msg, unsigned int flags);
\end{verbatim}
\end{quote}

{\sl DESCRIPTION}
\begin{quote}
	WARNING: This is a BSD man page. As of Linux 0.99.11, {\tt recvmsg}
	was not implemented.

	{\tt recvfrom} and {\tt recvmsg} are used to receive messages from a
	socket, and may be used to receive data on a socket whether or not
	it is connection-oriented.

	If from is non-nil, and the socket is not connection-oriented, the
	source address of the message is filled in. {\tt fromlen} is a
	value-result parameter, initialized to the size of the buffer
	associated with from, and modified on return to indicate the actual
	size of the address stored there.

	The {\tt recv} call is normally used only on a connected socket (see
	{\tt connect(2)}) and is identical to {\tt recvfrom} with a nil from
	parameter. As it is redundant, it may not be supported in future
	releases.

	All three routines return the length of the message on successful
	completion. If a message is too long to fit in the supplied buffer,
	excess bytes may be discarded depending on the type of socket the
	message is received from (see {\tt socket(2)}).

	If no messages are available at the socket, the receive call waits
	for a message to arrive, unless the socket is nonblocking (see {\tt
	fcntl(2)}) in which case the value -1 is returned and the external
	variable errno set to {\tt EWOULDBLOCK}. The receive calls normally
	return any data available, up to the requested amount, rather than
	waiting for receipt of the full amount requested; this behavior is
	affected by the socket-level options {\tt SO\_RCVLOWAT} and {\tt SO\_RCVTIMEO}
	described in {\tt getsockopt(2)}.

	The {\tt select(2)} call may be used to determine when more data
	arrive.

 	The flags argument to a recv call is formed by or'ing one
 	or more of the values:\\
	\begin{tabular}{lp{8cm}}
		{\tt MSG\_OOB}	&
			process out-of-band data\\
		{\tt MSG\_PEEK}	&
			peek at incoming message\\
		{\tt MSG\_WAITALL}	&
			wait for full request or error
	\end{tabular}

	The {\tt MSG\_OOB} flag requests receipt of out-of-band data that
	would not be received in the normal data stream. Some protocols
	place expedited data at the head of the normal data queue, and thus
	this flag cannot be used with such protocols. The {\tt MSG\_PEEK}
	flag causes the receive operation to return data from the beginning
	of the receive queue without removing that data from the queue.
	Thus, a subsequent receive call will return the same data. The {\tt
	MSG\_WAITALL} flag requests that the operation block until the full
	request is satisfied. However, the call may still return less data
	than requested if a signal is caught, an error or disconnect occurs,
	or the next data to be received is of a different type than that
	returned.

	The {\tt recvmsg} call uses a {\tt msghdr} structure to minimize the
	number of directly supplied parameters. This structure has the
	following form, as defined in {\tt sys/socket.h}:
	\begin{verbatim}
	struct msghdr {
	    caddr_t msg_name; /* optional address */
	    u_int msg_namelen; /* size of address */
	    struct iovec *msg_iov; /* scatter/gather array */
	    u_int msg_iovlen; /* # elements in msg_iov */
	    caddr_t msg_control; /* ancillary data, see below */
	    u_int msg_controllen; /* ancillary data buffer len */
	    int msg_flags; /* flags on received message */
	};
	\end{verbatim}

	Here {\tt msg\_name} and {\tt msg\_namelen} specify the destination
	address if the socket is unconnected; {\tt msg\_name} may be given
	as a null pointer if no names are desired or required. {\tt
	msg\_iov} and {\tt msg\_iovlen} describe scatter gather locations,
	as discussed in {\tt read(2)}. {\tt msg\_control}, which has length
	{\tt msg\_controllen}, points to a buffer for other protocol control
	related messages or other miscellaneous ancillary data. The messages
	are of the form:\\
	\begin{verbatim}
	struct cmsghdr {
	    u_int cmsg_len; /* data byte count, including hdr */
	    int cmsg_level; /* originating protocol */
	    int cmsg_type; /* protocol-specific type */
	    /* followed by u_char cmsg_data[]; */
	};
	\end{verbatim}

	As an example, one could use this to learn of changes in the
	data-stream in XNS/SPP, or in ISO, to obtain user-connection-request
	data by requesting a {\tt recvmsg} with no data buffer provided
	immediately after an accept call.

	Open file descriptors are now passed as ancillary data for {\tt AF\_UNIX}
	domain sockets, with {\tt cmsg\_level} set to {\tt SOL\_SOCKET} and {\tt
	cmsg\_type} set to {\tt SCM\_RIGHTS}.

	The {\tt msg\_flags} field is set on return according to the message
	received. {\tt MSG\_EOR} indicates end-of-record; the data returned
	completed a record (generally used with sockets of type {\tt
	SOCK\_SEQPACKET}). {\tt MSG\_TRUNC} indicates that the trailing
	portion of a datagram was discarded because the datagram was larger
	than the buffer supplied. {\tt MSG\_CTRUNC} indicates that some
	control data were discarded due to lack of space in the buffer for
	ancillary data. {\tt MSG\_OOB} is returned to indicate that
	expedited or out-of-band data were received.
\end{quote}

{\sl RETURN VALUES}
\begin{quote}
 	These calls return the number of bytes received, or -1 if
 	an error occurred.
\end{quote}

{\sl ERRORS}
\begin{quote}
\begin{tabular}{lp{8cm}}
	{\tt EBADF} &
		The argument s is an invalid descriptor.\\
	{\tt ENOTCONN} &
		The socket is associated with a connection-oriented protocol
		and has not been connected (see {\tt connect(2)} and {\tt
accept(2)}).\\

	{\tt ENOTSOCK} &
		The argument {\tt s} does not refer to a socket.\\
	{\tt EWOULDBLOCK} &
		The socket is marked non-blocking, and the receive
		operation would block, or a receive timeout had
		been set, and the timeout expired before data were
		received.\\
	{\tt EINTR} &
		The receive was interrupted by delivery of a signal before any data
		were available.\\
	{\tt EFAULT} &
		The receive buffer pointer(s) point outside the process's address
space.
\end{tabular}
\end{quote}

{\sl HISTORY}
\begin{quote}
 	These function calls appeared in BSD 4.2.
\end{quote}

{\sl SEE ALSO}
\begin{quote}
	{\tt fcntl(2)}, {\tt read(2)}, {\tt select(2)}, {\tt getsockopt(2)},
	{\tt socket(2)}
\end{quote}
\end{english}

%%%%%%%%%%%%%%%
\subsection{Manuel de {\tt select()}}
\begin{english}
{\sl SELECT(2)}\\

{\sl NAME}
\begin{quote}
	{\tt select} - synchronous I/O multiplexing
\end{quote}

{\sl SYNOPSIS}
\begin{quote}
\begin{verbatim}
#include <sys/time.h>
#include <sys/types.h>
#include <unistd.h>

int select(int n, fd_set *readfds, fd_set *writefds,
       fd_set *exceptfds, struct timeval *timeout);

FD_CLR(int fd, fd_set *set);
FD_ISSET(int fd, fd_set *set);
FD_SET(int fd, fd_set *set);
FD_ZERO(fd_set *set);
\end{verbatim}
\end{quote}

{\sl DESCRIPTION}
\begin{quote}
	{\tt select} waits for a number of file descriptors to change status.

	Three independent sets of descriptors are watched. Those listed in
	{\tt readfds} will be watched to see if characters become available 
	for reading, those in {\tt writefds} will be watched to see if it is ok
	to immediately write on them, and those in {\tt exceptfds} will be
	watched for exceptions. On exit, the sets are modified in place 
	to indicate which descriptors actually changed status.

	Four macros are provided to manipulate the sets.
	\begin{itemize}
		\item {\tt FD\_ZERO} will clear a set.
		\item {\tt FD\_SET} add a given descriptor from a set.
		\item {\tt FD\_CLR} remove a given descriptor from a set.
		\item {\tt FD\_ISSET} tests to see if a descriptor is part
				of the set; this is useful after select returns.
	\end{itemize}

	{\tt n} is the highest-numbered descriptor in any of the three sets,
	plus 1.

	{\tt timeout} is an upper bound on the amount of time elapsed before 
	select returns. It may be zero, causing select to return
	immediately. If the timeout is {\tt NULL} (no timeout), select can block
	indefinitely.
\end{quote}

{\sl RETURN VALUE}
\begin{quote}
	On success, {\tt select} returns the number of descriptors contained in
	the descriptor sets, which may be zero if the timeout expires 
	before anything interesting happens. On error, -1 is returned, and
	{\tt errno} is set appropriately; the sets and timeout become undefined,
	so do not rely on their contents after an error.
\end{quote}

{\sl ERRORS}
\begin{quote}
\begin{tabular}{lp{8cm}}
	{\tt EBADF}	&
		An invalid file descriptor was given in one of the sets.\\
	{\tt EINTR}	&
		A non blocked signal was caught.\\
	{\tt EINVAL} &
		{\tt n} is negative.\\
	{\tt ENOMEM} &
		{\tt select} was unable to allocate memory for internal
		tables.
\end{tabular}
\end{quote}

{\sl NOTES}
\begin{quote}
	Some code calls {\tt select} with all three sets empty, {\tt n} zero,
	and a non-null {\tt timeout} as a fairly portable way to sleep with
	subsecond precision.

	On Linux, {\tt timeout} is modified to reflect the amount of time
	not slept; most other implementations do not do this. This causes
	problems both when Linux code which reads {\tt timeout} is 
	ported to other operating systems, and when code is ported to Linux
	that reuses a {\tt struct timeval} for multiple selects in a 
	loop without reinitializing it. Consider {\tt timeout} to be
	undefined after select returns.
\end{quote}

{\sl EXAMPLE}
\begin{quote}
\begin{verbatim}
#include <stdio.h>
#include <sys/time.h>
#include <sys/types.h>
#include <unistd.h>

int main(void)
{
    fd_set rfds;
    struct timeval tv;
    int retval;

    /* Watch stdin (fd 0) to see when it has input. */
    FD_ZERO(&rfds);
    FD_SET(0, &rfds);
    /* Wait up to five seconds. */
    tv.tv_sec = 5;
    tv.tv_usec = 0;

    retval = select(1, &rfds, NULL, NULL, &tv);
    /* Don't rely on the value of tv now! */

    if (retval)
        printf("Data is available now.\n");
        /* FD_ISSET(0, &rfds) will be true */
    else
        printf("No data within five seconds.\n");
    exit(0);
}
\end{verbatim}
\end{quote}

{\sl SEE ALSO}
\begin{quote}
	{\tt accept(2)}, {\tt connect(2)}, {\tt read(2)}, {\tt recv(2)},
	{\tt send(2)}, {\tt write(2)}
\end{quote}
\end{english}


%%%%%%%%%%%%%%%
\subsection{Manuel de {\tt send()} et {\tt sendto()}}
\begin{english}
{\sl SEND(2)}\\

{\sl NAME}
\begin{quote}
	{\tt send}, {\tt sendto}, {\tt sendmsg} - send a message from a socket
\end{quote}

{\sl SYNOPSIS}
\begin{quote}
\begin{verbatim}
#include <sys/types.h>
#include <sys/socket.h>

int send(int s, const void *msg, int len ,
         unsigned int flags);

int sendto(int s, const void *msg, int len,
           unsigned int flags, const struct sockaddr *to,
           int tolen);

int sendmsg(int s, const struct msghdr *msg ,
            unsigned int flags);
\end{verbatim}
\end{quote}

{\sl DESCRIPTION}
\begin{quote}
	WARNING: This is a BSD man page. As of Linux 0.99.11, {\tt sendmsg}
	was not implemented.

	{\tt send}, {\tt sendto}, and {\tt sendmsg} are used to transmit a
	message to another socket. Send may be used only when the socket
	is in a connected state, while {\tt sendto} and {\tt sendmsg} may
	be used at any time.

	The address of the target is given by to with tolen specifying its
	size. The length of the message is given by {\tt len}. If the
	message is too long to pass atomically through the underlying
	protocol, the error {\tt EMSGSIZE} is returned, and the message is
	not transmitted.

	No indication of failure to deliver is implicit in a send. Locally
	detected errors are indicated by a return value of -1.

	If no messages space is available at the socket to hold the 
	message to be transmitted, then send normally blocks, unless the
	socket has been placed in non-blocking I/O mode. The {\tt
	select(2)} call may be used to determine when it is possible to send
	more data.

	The flags parameter may include one or more of the following:
	\begin{verbatim}
		#define MSG_OOB        0x1  /* process out-of-band data */
		#define MSG_DONTROUTE  0x4  /* bypass routing, use direct
		                               interface */
	\end{verbatim}

	The flag {\tt MSG\_OOB} is used to send out-of-band data on sockets
	that support this notion (e.g. {\tt SOCK\_STREAM}); the
	underlying protocol must also support out-of-band data. {\tt
	MSG\_DONTROUTE} is usually used only by diagnostic or routing
	programs.

	See {\tt recv(2)} for a description of the {\tt msghdr} structure.
\end{quote}

{\sl RETURN VALUES}
\begin{quote}
	The  call  returns the number of characters sent, or -1 if an error
	occurred.
\end{quote}

{\sl ERRORS}
\begin{quote}
\begin{tabular}{lp{8cm}}
	{\tt EBADF} &
		An invalid descriptor was specified.\\
	{\tt ENOTSOCK} &
		The argument s is not a socket.\\
	{\tt EFAULT} &
		An invalid user space address was specified for a
		parameter.\\
	{\tt EMSGSIZE} &
		The socket requires that message be sent atomically,
		and the size of the message to be sent made
		this impossible.\\
	{\tt EWOULDBLOCK} &
		The socket is marked non-blocking and the
		requested operation would block.\\
	{\tt ENOBUFS} &
		The system was unable to allocate an internal
		buffer. The operation may succeed when buffers
		become available.\\
	{\tt ENOBUFS} &
		The output queue for a network interface was full.
		This generally indicates that the interface has
		stopped sending, but may be caused by transient
		congestion.
\end{tabular}
\end{quote}

{\sl HISTORY}
\begin{quote}
	These function calls appeared in BSD 4.2.
\end{quote}

{\sl SEE ALSO}
\begin{quote}
	{\tt fcntl(2)}, {\tt recv(2)}, {\tt select(2)}, {\tt getsockopt(2)},
	{\tt socket(2)}, {\tt write(2)}
\end{quote}
\end{english}

%%%%%%%%%%%%%%%
\subsection{Manuel de {\tt signal()}}
\begin{english}
{\sl SIGNAL(2)}\\

{\sl NAME}
\begin{quote}
	{\tt signal} - ANSI C signal handling.
\end{quote}

{\sl SYNOPSIS}
\begin{quote}
\begin{verbatim}
#include <signal.h>

void (*signal(int signum, void (*handler)(int)))(int);
\end{verbatim}
\end{quote}

{\sl DESCRIPTION}
\begin{quote}
	The {\tt signal} system call installs a new signal handler for signal
	{\tt signum}. The signal handler is set to handler which may be a user
	specified function, or one of the following:

	\begin{tabular}{lp{8cm}}
		{\tt SIG\_IGN} & Ignore the signal.\\
		{\tt SIG\_DFL} & Reset the signal to its default behavior.
	\end{tabular}
\end{quote}

{\sl RETURN VALUE}
\begin{quote}
	{\tt signal} returns the previous value of the signal handler, or
	{\tt SIG\_ERR} on error.
\end{quote}

{\sl NOTES}
\begin{quote}
	Signal handlers cannot be set for {\tt SIGKILL} or {\tt SIGSTOP}.

	Unlike BSD systems, signals under Linux are reset to their default
	behavior when raised.

	If you're confused by the prototype at the top of this manpage,
	it may help to see it separated out thus:
	\begin{verbatim}
		typedef void (*sighandler_t)(int);
		sighandler_t signal(int signum, sighandler_t handler);
	\end{verbatim}
\end{quote}

{\sl CONFORMING TO}
\begin{quote}
	ANSI C
\end{quote}

{\sl SEE ALSO}
\begin{quote}
	{\tt kill(1)}, {\tt kill(2)}, {\tt killpg(2)}, {\tt pause(2)}, {\tt
raise(3)},
	{\tt sigaction(2)}, {\tt signal(7)}, {\tt sigsetops(3)}, {\tt
sigvec(2)}
\end{quote}
\end{english}

%%%%%%%%%%%%%%%
\subsection{Manuel de {\tt socket()}}
\begin{english}
{\sl SOCKET(2)}\\

{\sl NAME}
\begin{quote}
	{\tt socket} - create an endpoint for communication
\end{quote}

{\sl SYNOPSIS}
\begin{quote}
\begin{verbatim}
#include <sys/types.h>
#include <sys/socket.h>

int socket(int domain, int type, int protocol);
\end{verbatim}
\end{quote}

{\sl DESCRIPTION}
\begin{quote}
	{\tt socket} creates an endpoint for communication and returns a
	descriptor.

	The {\tt domain} parameter specifies a communications domain within 
	which communication will take place; this selects the protocol
	family which should be used. These families are defined in the
	include file {\tt sys/socket.h}. The currently understood formats are:
	\begin{quote}
	\begin{tabular}{lp{5cm}}
		{\tt AF\_UNIX}		&	(UNIX internal protocols)\\
		{\tt AF\_INET}		&	(ARPA Internet protocols)\\
		{\tt AF\_ISO}		&	(ISO protocols)\\
		{\tt AF\_NS}			&	(Xerox Network Systems protocols)\\
		{\tt AF\_IMPLINK}	&	(IMP ``host at IMP'' link layer)
	\end{tabular}
	\end{quote}

	The socket has the indicated type, which specifies the semantics of
	communication. Currently defined types are:
	\begin{quote}
		{\tt SOCK\_STREAM}\\
		{\tt SOCK\_DGRAM}\\
		{\tt SOCK\_RAW}\\
		{\tt SOCK\_SEQPACKET}\\
		{\tt SOCK\_RDM}
	\end{quote}
	
	A {\tt SOCK\_STREAM} type provides sequenced, reliable, two-way connection
	based byte streams. An out-of-band data transmission mechanism 
	may be supported. A {\tt SOCK\_DGRAM} socket supports datagrams
	(connectionless, unreliable messages of a fixed (typically small)
	maximum length). A {\tt SOCK\_SEQPACKET} socket may provide a sequenced, 
	reliable, two-way connection-based data transmission path for
	datagrams of fixed maximum length; a consumer may be required to 
	read an entire packet with each read system call. This facility is
	protocol specific, and presently implemented only for {\tt PF\_NS}. 
	{\tt SOCK\_RAW} sockets provide access to internal network protocols and 
	interfaces. The types {\tt SOCK\_RAW}, which is available only to the
	super-user, and {\tt SOCK\_RDM}, which is planned, but not yet 
	implemented, are not described here.

	The protocol specifies a particular protocol to be used with
	the socket. Normally only a single protocol exists to support a
	particular socket type within a given protocol family. However,
	it is possible that many protocols may exist, in which case a
	particular protocol must be specified in this manner. The protocol
	number to use is particular to the ``communication domain'' in
	which communication is to take place; see {\tt protocols(5)}.

	Sockets of type {\tt SOCK\_STREAM} are full-duplex byte streams, similar 
	to pipes. A stream socket must be in a connected state before any
	data may be sent or received on it. A connection to another
	socket is created with a {\tt connect(2)} call. Once connected, data 
	may be transferred using {\tt read(2)} and {\tt write(2)} calls or some
	variant of the {\tt send(2)} and {\tt recv(2)} calls. When a session has been 
	completed a {\tt close(2)} may be performed. Out-of-band data may also
	be transmitted as described in {\tt send(2)} and received as
	described in {\tt recv(2)}.

	The communications protocols used to implement a
	{\tt SOCK\_STREAM} insure that data is not lost or duplicated. If a
	piece of data for which the peer protocol has buffer space cannot be
	successfully transmitted within a reasonable length of time, 
	then the connection is considered broken and calls will indicate an
	error with -1 returns and with {\tt ETIMEDOUT} as the specific
	code in the global variable {\tt errno}. The protocols optionally keep
	sockets warm by forcing transmissions roughly every minute in the
	absence of other activity. An error is then indicated if no 
	response can be elicited on an otherwise idle connection for a
	extended period (e.g. 5 minutes). A {\tt SIGPIPE} signal is raised 
	if a process sends on a broken stream; this causes naive processes,
	which do not handle the signal, to exit.

	{\tt SOCK\_SEQPACKET} sockets employ the same system calls as
	{\tt SOCK\_STREAM} sockets. The only difference is that {\tt read(2)} calls 
	will return only the amount of data requested, and any remaining in
	the arriving packet will be discarded.

	{\tt SOCK\_DGRAM} and {\tt SOCK\_RAW} sockets allow sending of datagrams to 
	correspondents named in {\tt send(2)} calls. Datagrams are generally
	received with {\tt recvfrom(2)}, which returns the next datagram with
	its return address.

	An {\tt fcntl(2)} call can be used to specify a process group to receive a
	{\tt SIGURG} signal when the out-of-band data arrives. It may also enable
	non-blocking I/O and asynchronous notification of I/O events via
	{\tt SIGIO}.

	The operation of sockets is controlled by socket level options. 
	These options are defined in the file {\tt sys/socket.h}. 
	{\tt setsockopt(2)} and {\tt getsockopt(2)} are used to set and get options,
	respectively.
\end{quote}

{\sl RETURN VALUES}
\begin{quote}
	A  -1 is returned if an error occurs, otherwise the return
	value is a descriptor referencing the socket.
\end{quote}

{\sl ERRORS}
\begin{quote}
\begin{tabular}{lp{8cm}}
	{\tt EPROTONOSUPPORT}	&
		The protocol type or the specified protocol is not supported within
		this domain.\\
	{\tt EMFILE}	&
		The per-process descriptor table is full.\\
	{\tt ENFILE}	&
		The system file table is full.\\
	{\tt EACCESS}	&
		Permission to create a socket of the specified type and/or protocol
		is denied.\\
	{\tt ENOBUFS}	&
		Insufficient buffer space is available. The socket cannot be created
		until sufficient resources are freed.
	\end{tabular}
\end{quote}

{\sl HISTORY}
\begin{quote}
	The socket function call appeared in BSD 4.2.
\end{quote}

{\sl SEE ALSO}
\begin{quote}
	{\tt accept(2)}, {\tt bind(2)}, {\tt connect(2)}, {\tt
	getprotoent(3)}, {\tt getsockname(2)}, {\tt getsockopt(2)}, {\tt
	ioctl(2)}, {\tt listen(2)}, {\tt read(2)}, {\tt recv(2)}, {\tt
	select(2)}, {\tt send(2)}, {\tt shutdown(2)}, {\tt socketpair(2)},
	{\tt write(2)}

	<<~An Introductory 4.3 BSD Interprocess Communication Tutorial~>> is
	reprinted in {\Unix} Programmer's Supplementary Documents Volume 1

	<<~BSD Interprocess Communication Tutorial~>> is reprinted in {\Unix}
	Programmer's Supplementary Documents Volume 1
\end{quote}
\end{english}

%%%%%%%%%%%%%%%
\subsection{Manuel de {\tt write()}}
\begin{english}
{\sl WRITE(2)}\\

{\sl NAME}
\begin{quote}
	write - write to a file descriptor
\end{quote}

{\sl SYNOPSIS}
\begin{quote}
\begin{verbatim}
#include <sys/types.h>
#include <unistd.h>

size_t write(int fd, const char *buf, size_t count);
\end{verbatim}
\end{quote}

{\sl DESCRIPTION}
\begin{quote}
	{\tt write} writes up to {\tt count} bytes from file descriptor
	{\tt fd} from the buffer starting at {\tt buf}.
\end{quote}

{\sl RETURN VALUE}
\begin{quote}
	On success, the number of bytes written are returned (zero indicates
	 nothing was written). On error, -1 (or {\tt MAXINT}, as {\tt
	size\_t} is unsigned) is returned, and {\tt errno} is set
	appropriately.
\end{quote}

{\sl ERRORS}
\begin{quote}
	\begin{tabular}{lp{8cm}}
		{\tt EBADF}	&
			{\tt fd} is not a valid file descriptor or is not open for writing.\\
		{\tt EINVAL}	&
			{\tt fd} is attached to an object which is unsuitable for writing.\\
		{\tt EFAULT}	&
			{\tt buf} is outside your accessible address space.\\
		{\tt EPIPE}	&
			{\tt fd} is connected to a pipe or socket whose reading end is closed. When
			this happens the writing process will receive a {\tt SIGPIPE} signal; if
			it catches, blocks or ignores this the error {\tt EPIPE} is returned.\\
		{\tt EAGAIN}	&
			Non-blocking I/O has been selected using {\tt O\_NONBLOCK} and there was no
			room in the pipe or socket connected to {\tt fd} to write the data
			immediately.\\
		{\tt EINTR}	&
			The call was interrupted by a signal before any data was written.\\
		{\tt ENOSPC}	&
			The device containing the file referred to by {\tt fd} has no room for the
			data.\\
	\end{tabular}

	Other errors may occur, depending on the object connected to {\tt fd}.
\end{quote}

{\sl CONFORMING TO}
\begin{quote}
	SVID, AT\&T, POSIX, X/OPEN, BSD 4.3
\end{quote}

{\sl SEE ALSO}
\begin{quote}
	{\tt open(2)}, {\tt read(2)}, {\tt fcntl(2)}, {\tt close(2)},
	{\tt lseek(2)}, {\tt select(2)}, {\tt ioctl(2)}, {\tt fwrite(3)}.
\end{quote}
\end{english}

%%%%%%%%%%%%%%%
\section{Pages du manuel de la section 8 (commandes administrateur)}

%%%%%%%%%%%%%%%
\subsection{Manuel de {\tt inetd}}
\begin{english}
{\sl INETD(8)}\\

{\sl NAME}
\begin{quote}
	{\tt inetd} - internet ``super-server''
\end{quote}

{\sl SYNOPSIS}
\begin{quote}
	{\tt inetd [-d] [configuration file]}
\end{quote}

{\sl DESCRIPTION}
\begin{quote}
	Inetd should be run at boot time by {\tt /etc/rc.local} (see {\tt
	rc(8)}). It then listens for connections on certain internet
	sockets. When a connection is found on one of its sockets, it
	decides what service the socket corresponds to, and invokes a
	program to service the request. After the program is finished, it
	continues to listen on the socket (except in some cases which will
	be described below). Essentially, inetd allows running one daemon
	to invoke several others, reducing load on the system.

	The option available for {\tt inetd}:
	\begin{description}
		\item[{\tt -d}] Turns on debugging.
	\end{description}

	Upon execution, {\tt inetd} reads its configuration information from
	a configuration file which, by default, is {\tt /etc/inetd.conf}.
	There must be an entry for each field of the configuration file,
	with entries for each field separated by a tab or a space.  Comments
	are denoted by a ``{\tt \#}'' at the beginning of a line.  There
	must be an entry for each field.  The fields of the configuration
	file are as follows:
	\begin{itemize}
		\item service name
		\item socket type
		\item protocol
 		\item wait/nowait[.max]
		\item user[.group]
		\item server program
		\item server program arguments
	\end{itemize}

	To specify an Sun-RPC based service, the entry would contain these
	fields.
	\begin{itemize}
		\item service name/version
		\item socket type
		\item rpc/protocol
		\item wait/nowait[.max]
		\item user[.group]
		\item server program
		\item server program arguments
	\end{itemize}

	The service-name entry is the name of a valid service in the
	file {\tt /etc/services}. For ``internal'' services (discussed
	below), the service name must be the official name of the service
	(that is, the first entry in {\tt /etc/services}). When used to
	specify a Sun-RPC based service, this field is a valid RPC service
	name in the file {\tt /etc/rpc}. The part on the right of the ``{\tt
	/}'' is the RPC version number. This can simply be a single numeric
	argument or a range of versions. A range is bounded by the low
	version to the high version - ``{\tt rusers/1-3}''.

	The socket-type should be one of ``{\tt stream}'', ``{\tt dgram}'',
	``{\tt raw}'', ``{\tt rdm}'', or ``{\tt seqpacket}'', depending on
	whether the socket is a stream, datagram, raw, reliably delivered
	message, or sequenced packet socket.

	The protocol must be a valid protocol as given in {\tt /etc/protocols}.
	Examples might be ``{\tt tcp}'' or ``{\tt udp}''. Rpc based services are
	specified with the ``{\tt rpc/tcp}'' or ``{\tt rpc/udp}'' service type.

	The {\tt wait}/{\tt nowait} entry is applicable to datagram sockets
	only (other sockets should have a ``{\tt nowait}'' entry in this
	space). If a datagram server connects to its peer, freeing the
	socket so {\tt inetd} can received further messages on the socket,
	it is said to be a ``multi-threaded'' server, and should use the
	``{\tt nowait}'' entry. For datagram servers which process all
	incoming datagrams on a socket and eventually time out, the server
	is said to be ``single-threaded'' and should use a ``{\tt wait}''
	entry. {\tt comsat(8)} ({\tt biff(1)}) and {\tt talkd(8)} are both
	examples of the latter type of datagram server. {\tt tftpd(8)} is
	an exception; it is a datagram server that establishes
	pseudo-connections. It must be listed as ``{\tt wait}'' in order to
	avoid a race; the server reads the first packet, creates a new
	socket, and then forks and exits to allow {\tt inetd} to check for
	new service requests to spawn new servers. The optional ``{\tt
	max}'' suffix (separated from ``{\tt wait}'' or ``{\tt nowait}'' by
	a dot) specifies the maximum number of server instances that may be
	spawned from {\tt inetd} within an interval of 60 seconds. When
	omitted, ``{\tt max}'' defaults to 40.

	The user entry should contain the user name of the user as whom the
	server should run. This allows for servers to be given less
	permission than root. An optional group name can be specified by
	appending a dot to the user name followed by the group name. This
	allows for servers to run with a different (primary) group id than
	specified in the password file. If a group is specified and user is
	not root, the supplementary groups associated with that user will
	still be set.

	The server-program entry should contain the pathname of the program
	which is to be executed by {\tt inetd} when a request is found on
	its socket. If {\tt inetd} provides this service internally, this entry
	should be ``internal''.

	The server program arguments should be just as arguments normally
	are, starting with {\tt argv[0]}, which is the name of the program. 
	If the service is provided internally, the word ``internal'' should
	take the place of this entry.

	{\tt inetd} provides several ``trivial'' services internally by use
	of routines within itself. These services are ``{\tt echo}'',
	``{\tt discard}'', ``{\tt chargen}'' (character generator), ``{\tt
	daytime}'' (human readable time), and ``{\tt time}'' (machine
	readable time, in the form of the number of seconds since midnight,
	January 1, 1900). All of these services are tcp based. For details
	of these services, consult the appropriate RFC from the Network
	Information Center.

	{\tt inetd} rereads its configuration file when it receives a hangup
	signal, {\tt SIGHUP}. Services may be added, deleted or modified
	when the configuration file is reread. {\tt inetd} creates a file
	{\tt /etc/inetd.pid} that contains its process identifier.
\end{quote}

{\sl SEE ALSO}
\begin{quote}
	{\tt comsat(8)}, {\tt fingerd(8)}, {\tt ftpd(8)}, {\tt rexecd(8)},
	{\tt rlogind(8)}, {\tt rshd(8)}, {\tt telnetd(8)}, {\tt tftpd(8)}
\end{quote}

{\sl HISTORY}
\begin{quote}
	The {\tt inetd} command appeared in 4.3BSD. Support for Sun-RPC based
	services is modelled after that provided by Sun-OS 4.1.
\end{quote}
\end{english}

%%%%%%%%%%%%%%%
\subsection{Manuel de la commande {\tt netstat}}
\begin{english}
{\sl NETSTAT(8)}\\

{\sl NAME}
\begin{quote}
	{\tt netstat} - Display active network connections
\end{quote}

{\sl SYNOPSIS}
\begin{quote}
\begin{verbatim}
netstat [[-a | [-t | -u | -w]] [-n | -o]  -x] [-c]

netstat -i [-a] [-c]

netstat -r [-c] [-n]

netstat -v
\end{verbatim}
\end{quote}

{\sl DESCRIPTION}
\begin{quote}
	{\tt netstat} displays the status of network connections on
	either {\tt TCP}, {\tt UDP}, {\tt RAW} or {\Unix} sockets to 
	the system. By default, {\tt netstat} only displays status on
	active sockets which are not in the {\tt LISTEN} state (i.e. 
	connections to active processes). To obtain information about the
	kernel routing table, {\tt netstat} may be invoked with the option
	{\tt -r}

	{\tt netstat}'s display includes the following information for each
	socket:
	\begin{description}
		\item[{\sl Proto}]\mbox{}\\
			The protocol (either {\tt TCP} or {\tt UDP}) used by the socket.
		\item[{\sl Recv-Q}]\mbox{}\\
			The count of bytes not copied by the user program connected
			to this socket.
		\item[{\sl Send-Q}]\mbox{}\\
			The count of bytes not acknoledged by the remote host.
		\item[{\sl Local Address}]\mbox{}\\
			The local address (local hostname) and port number of the
			socket. Unless the {\tt -n} switch is given, the socket address
			is resolved to its canonical hostname, and the port number
			is translated into the corresponding service name.
		\item[{\sl Foreign Address}]\mbox{}\\
			The remote address (remote hostname) and port number of the
			socket. As with the local {\tt address:port}, the {\tt -n} switch
			turns off hostname and service name resolution.
		\item[{\sl (State)}]\mbox{}\\
			The state of the socket. Since there are no states in {\tt RAW}
			and usually no states used in {\tt UDP}, this row may be left
			blank. Normally this can be one of several values:
			\begin{description}
				\item[{\tt ESTABLISHED}]\mbox{}\\
					The socket has an established connection.
				\item[{\tt SYN\_SENT}]\mbox{}\\
					The socket is actively attempting to establish a
					connection.
				\item[{\tt SYN\_RECV}]\mbox{}\\
					The connection is being initialized.
				\item[{\tt FIN\_WAIT1}]\mbox{}\\
					The socket is closed, and the connection is shutting down.
				\item[{\tt FIN\_WAIT2}]\mbox{}\\
					Connection is closed, and the socket is waiting for
					a shutdown from the remote end.
				\item[{\tt TIME\_WAIT}]\mbox{}\\
					The socket is waiting after close for remote shutdown
					retransmission.
				\item[{\tt CLOSED}]\mbox{}\\
					The socket is not being used.
				\item[{\tt CLOSE\_WAIT}]\mbox{}\\
					The remote end has shut down, waiting for the
					socket to close.
				\item[{\tt LAST\_ACK}]\mbox{}\\
					The remote end shut down, and the socket is closed.
					Waiting for acknowledgement.
				\item[{\tt LISTEN}]\mbox{}\\
					The socket is listening for incoming connections.
				\item[{\tt CLOSING}]\mbox{}\\
					Both sockets are shut down but we still don't have
					all our data sent.
				\item[{\tt UNKNOWN}]\mbox{}\\
					The state of the socket is unknown.
				\end{description}
	\end{description}
	
	If {\tt netstat} is invoked with the option {\tt -o} , 
	additional information will be displayed behind the state
	info. These informations are shown like this: ``{\tt
	rx-retransmission byte count}'' ``{\tt tx-retransmission byte
	count}'' ``{\tt timer state}'' ``{\tt (time/backoff)}''. Timer state
	may now be either on or off. The time (in seconds) being displayed
	is how long it will take the timer to expire. All these options are
	subject to be removed in later releases of the {\tt NET} software.

	Being invoked with the option {\tt -x} , {\tt netstat} displays a
	list of all active {\Unix} internal communication sockets.

	{\tt netstat}'s display includes the following information for each
	socket:
	\begin{description}
		\item[{\sl Proto}]\mbox{}\\
			The protocol (usually {\Unix}) used by the socket.
		\item[{\sl RefCnt}]\mbox{}\\
			The reference count (i.e. attached processes via this
			socket).
		\item[{\sl Flags}]\mbox{}\\
			The only displayed flag is {\tt SO\_ACCEPTON} (displayed as {\tt ACC})
			otherwise left blank. {\tt SO\_ACCECPTON} is used on unconnected
			sockets if their corresponding processes are waiting for a
			connect request. The other flags are not of normal interest and
			not displayed.
		\item[{\sl Type}]\mbox{}\\
			There are several types of socket access:
			\begin{description}
				\item[{\tt SOCK\_DGRAM}]\mbox{}\\
					The socket is used in Datagram (connectionless)
					mode.
				\item[{\tt SOCK\_STREAM}]\mbox{}\\
					This is a stream (connection) socket.
				\item[{\tt SOCK\_RAW}]\mbox{}\\
					The socket is used as a raw socket.
				\item[{\tt SOCK\_RDM}]\mbox{}\\
					This one serves reliably-delivered messages.
				\item[{\tt SOCK\_SEQPACKET}]\mbox{}\\
					This is a sequential packet socket.
				\item[{\tt SOCK\_PACKET}]\mbox{}\\
					{\tt RAW} interface access socket.
				\item[{\tt UNKNOWN}]\mbox{}\\
					Who ever knows, what the future will bring us -
					just fill in here :-)
			\end{description}
		\item[{\sl State}]\mbox{}\\
			This field will contain one of the following Keywords:
			\begin{description}
				\item[{\tt FREE}]\mbox{}\\
					The socket is not allocated
				\item[{\tt LISTENING}]\mbox{}\\
					The socket is listening for a connection request.
				\item[{\tt UNCONNECTED}]\mbox{}\\
					The socket is not connected to another one.
				\item[{\tt CONNECTING}]\mbox{}\\
					The socket is about to establish a connection.
				\item[{\tt CONNECTED}]\mbox{}\\
					The socket is connected.
				\item[{\tt DISCONNECTING}]\mbox{}\\
					The socket is disconnecting.
				\item[{\tt UNKNOWN}]\mbox{}\\
					This state should never happen.
			\end{description}
		\item[{\sl Path}]\mbox{}\\
			This displays the path name as which the corresponding
			processes attached to the socket.
	\end{description}

	The network routing table (invoked with {\tt netstat -r}) shows up
	the following information:
	\begin{description}
		\item[{\sl Destination net/address}]\mbox{}\\
			The destination adress of a resolved host or hand-entered
			network is displayed. Unless the option {\tt -n} is given, the
			hosts or nets are resolved. An entry named ``{\tt default}'' shows
			up the default route for the kernel.
		\item[{\sl Gateway address}]\mbox{}\\
			If there is no Asterisk (``{\tt *}'') displayed - any data will be
			routed to the dedicated gateway.
		\item[{\sl Flags}]\mbox{}\\
			Possible routeing flags are:\\
			\begin{tabular}{lp{8cm}}
				{\tt U} & This route is useable \\
				{\tt G} & Destination is a gateway \\
				{\tt H} & Destination is a Host entry \\
				{\tt R} & Route will be reinstated after time-out \\
				{\tt D} & This one is created dynamically (by redirection) \\
				{\tt M} & This one is modified dynamically (by redirection)
			\end{tabular}
		\item[{\sl RefCnt}]\mbox{}\\
			Reference count for this route
		\item[{\sl Use}]\mbox{}\\
			How many times this route was used yet
		\item[{\sl Iface}]\mbox{}\\
			This is the name of the Interface, where this route
			belongs to.
	\end{description}

	The device statistics table (invoked with {\tt netstat -i})
	displays information about the interfaces:
	\begin{description}
		\item[{\sl Iface}]\mbox{}\\
			The name this interface.
		\item[{\tt MTU}]\mbox{}\\
			Maximum size for transmission on this interface. This
			should be the size of data transferred on this interface
			without interface specific headers.
		\item[{\tt RX-OK}]\mbox{}\\
			error free received packets for this interface.
		\item[{\tt RX-ERR}]\mbox{}\\
			buggy received packets.
		\item[{\tt RX-DRP}]\mbox{}\\
			dropped received packets (due to memory lack ?).
		\item[{\tt RX-OVR}]\mbox{}\\
			packets that we were unable to receive that fast way.
		\item[{\tt TX-OK}]\mbox{}\\
			error free transmitted packets for this interface.
		\item[{\tt TX-ERR}]\mbox{}\\
			buggy transmitted packets.
		\item[{\tt TX-DRP}]\mbox{}\\
			dropped transmitteded packets.
		\item[{\tt TX-OVR}]\mbox{}\\
			packets that we were unable to transmit.
		\item[{\sl Flags}]\mbox{}\\
			The following flags may occur on the given interfaces:\\
			\begin{tabular}{lp{8cm}}
			{\tt A} & This interface will receive all Multicast adresses.\\
			{\tt B} & Broadcasts are ok here.\\
			{\tt D} & Debugging is turned on.\\
			{\tt L} & Interface is a loopback device.\\
			{\tt M} & All packets are received (Promisc-Mode).\\
			{\tt N} & Trailers are avoided.\\
			{\tt O} & No Addres Resolution Protocol on this Interface.\\
			{\tt P} & Interface is a Point-to-Point connection.\\
			{\tt R} & Interface is running.\\
			{\tt U} & Interface is up.
			\end{tabular}
	\end{description}

	Options:\\
	\begin{tabular}{lp{8cm}}
	{\tt -a}	&	Display information about all internet sockets,
					i.e. {\tt TCP}, {\tt UDP}, {\tt RAW} and {\Unix}
					including those sockets that are listening only.\\
	{\tt -i}	&	Show network devices statistics.\\
	{\tt -c}	&	Generate a continuous listing of network status:
					network status is displayed every second until the
					program is interrupted.\\
	{\tt -n}	&	Causes {\tt netstat} not to resolve hostnames and service
					names when displaying remote and local address and
					port information.\\
	{\tt -o}	&	Display timer states, expiration times and backoff
					state.\\
	{\tt -r}	&	Display kernel routing table.\\
	{\tt -t}	&	Display information about {\tt TCP} sockets only, including
					those that are listening.\\
	{\tt -u}	&	Display information about {\tt UDP} sockets only.\\
	{\tt -v}	&	Print version information.\\
	{\tt -w}	&	Display information about raw sockets.\\
	{\tt -x}	&	Display information about {\Unix} domain sockets.
	\end{tabular}
\end{quote}

{\sl FILES}
\begin{quote}
	{\tt /etc/services} -- The services translation file\\

	{\tt /proc/net/socket} -- devices information\\

	{\tt /proc/net/raw} -- {\tt RAW} socket information\\

	{\tt /proc/net/route} -- Kernel routing information\\

	{\tt /proc/net/tcp} -- {\tt TCP} socket information\\

	{\tt /proc/net/udp} -- {\tt UDP} socket information\\

	{\tt /proc/net/unix} -- {\Unix} domain socket information
\end{quote}

{\sl BUGS}
\begin{quote}
	Occasionally strange information may appear if a socket changes
	as it is viewed. This is unlikely to occur.
\end{quote}

{\sl AUTHORS}
\begin{quote}
	The netstat user interface was written by Fred Baumgarten\footnote{{\tt
	dc6iq@insu1.etec.uni-karlsruhe.de}} the man page basically by Matt
	Welsh\footnote{{\tt mdw@tc.cornell.edu}}. It was updated by Alan
	Cox\footnote{{\tt Alan.Cox@linux.org}} but could do with a bit more
	work.
\end{quote}
\end{english}

%%%%%%%%%%%%%%%%%%%%%%%%%%%%%%%%%%%%%%%%%%%%%%%%%%%%%%%%%%%%%%%%%%%%%
%
% Fin Annexes
%
\end{appendix}