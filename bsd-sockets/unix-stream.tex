%%%%%%%%%%%%%%%%%%%%%%%%%%%%%%%%%%%%%%%%%%%%%%%%%%%%%%%%%%%%%%%%%%%%%
%
% Les Unix Stream Sockets
%
\chapter{\label{unstream}Les {\Unix} Stream Sockets}

%%%%%%%%%%%%%%%%%%%%%%%%%%%%%%%%%%%%%%%%%%%%%%%%%%%%%%%%%%%%%%%%%%%%%
\section{Rappels}

Les {\sl {\Unix} Stream Sockets} sont utilis{\'e}es pour permettre {\`a} deux
processus {\bf sur une m{\^e}me machine}, d'{\'e}changer des informations.

Nous introduirons donc ici une nouvelle famille d'adresses~: la famille <<~{\tt AF\_UNIX}~>>.

Ce type de canal de communication a un comportement quasi similaire aux
{\sl Internet Stream Sockets} (cf. chapitre ~\ref{instream}). On
retrouvera les caract{\'e}ristiques essentielles suivantes~:
\begin{itemize}
	\item	les canaux sont bidirectionnels,
	\item	les messages sont bien d{\'e}livr{\'e}s au processus destination,
	\item	les donn{\'e}es arrivent sans erreur,
	\item	l'ordre des messages est pr{\'e}serv{\'e}.
\end{itemize}

De la m{\^e}me fa\c{c}on, ces canaux de communications ne garantissent pas la
coh{\'e}rence au niveau de la longueur des messages. Une op{\'e}ration de
lecture sur le canal de communication (appel syst{\`e}me {\tt read(2)} ou {\tt recv(2)}) peut lire un volume de
donn{\'e}es inf{\'e}rieur ou sup{\'e}rieur {\`a} celui qui a {\'e}t{\'e} effectu{\'e} lors de l'op{\'e}ration
d'{\'e}criture d'un seul message (appel syst{\`e}me {\tt write(2)} ou {\tt send(2)}).

Comme nous l'avons vu dans la section \ref{intsock-intro-fam-proto}, la
famille {\tt AF\_UNIX} n'admet qu'un seul mode de connexion~: le mode
connect{\'e} ({\tt SOCK\_STREAM}) (cf. figure \ref{intsock-fig-famille-protocole}). Nous avons vu que ce type de
socket est plus avantageux que la notion de {\sl loopback} dans IP\footnote{adresse IP sp{\'e}ciale permettant
{\`a} la machine de se r{\'e}f{\'e}rencer elle-m{\^e}me}. Comme on peut le voir sur la figure
\ref{intsock-fig-loopback-local},
\begin{itemize}
	\item	lorsqu'on utilise la famille d'adresses {\tt AF\_UNIX}, les donn{\'e}es
		sont directement envoy{\'e}es vers le processus destination au niveau
		4 du mod{\`e}le OSI (couche {\sl transport}),
	\item	lorsqu'on utilise la famille d'adresses {\tt AF\_INET}, les donn{\'e}es
		doivent descendre jusqu'au niveau 2 du mod{\`e}le OSI (couche
		{\sl liaison}) pour se rendre compte qu'il ne faut pas partir sur le
		support physique. Les donn{\'e}es devront donc traverser {\`a} nouveau toutes
		les couches interm{\'e}diaires (pour arriver jusqu'au niveau 4 (couche {\sl transport}).
\end{itemize}

La similitude entre les {\sl {\Unix} Stream Sockets} et les {\sl Internet Stream
Sockets}, est assez grande. Nous ne donnerons, dans toute la suite de ce
chapitre, que les diff{\'e}rences entre les deux.

Dans la section ~\ref{unstream-recap}, nous ferons un r{\'e}sum{\'e} de la s{\'e}quence
des appels syst{\`e}mes n{\'e}cessaires et un r{\'e}capitulatif des structures de donn{\'e}es
en montrant la diff{\'e}rence entre les {\sl {\Unix} Stream Sockets} et les
{\sl Internet Stream Sockets}.

%%%%%%%%%%%%%%%%%%%%%%%%%%%%%%%%%%%%%%%%%%%%%%%%%%%%%%%%%%%%%%%%%%%%%
\section{Attributs des sockets du domaine {\tt AF\_UNIX}}

Sachant que les communications autoris{\'e}es pour le domaine {\tt AF\_UNIX}
ne sont qu'intra-machine, beaucoup de champs que nous avons vus {\`a} la  cr{\'e}ation
d'un port de communications dans le chapitre ~\ref{instream} ne sont plus n{\'e}cessaires.

Pour les {\sl Internet Stream Sockets}, le programmeur se servait de la structure {\tt sockaddr\_in}
(cf. annexe \ref{ann-struct-sockaddrin}) pour sp{\'e}cifier les diff{\'e}rentes caract{\'e}ristiques de la socket.
Dans le cas des {\sl {\Unix} Stream Sockets}, la structure utilis{\'e}e est {\tt sockaddr\_un}.
Elle est d{\'e}finie dans le fichier {\tt sys$/$un.h} de la fa�on suivante~:
\begin{quote}
\begin{verbatim}
#include <sys/un.h>

struct sockaddr_un {
    short  sun_family;
    u_char sun_path[108];
}
\end{verbatim}
\end{quote}

Pour plus de pr{\'e}cisions sur cette structure, reportez vous {\`a} l'annexe \ref{ann-struct-sockaddrun}.

%%%%%%%%%%%%%%%%%%%%%%%%%%%%%%%%%%%%%%%%%%%%%%%%%%%%%%%%%%%%%%%%%%%%%
\section{\label{unstream-recap}R{\'e}capitulatif}

Le tableau ~\ref{unstream-syscall-recap} donne un r{\'e}capitulatif des diff{\'e}rents
appels syst{\`e}mes {\`a} utiliser afin d'{\'e}tablir un canal de communication
entre deux processus.

Les appels syst{\`e}mes marqu{\'e}s de <<~\dag{}~>> ou de <<~\ddag{}~>> ont des
arguments diff{\'e}rents de ceux utilis{\'e}s pour les {\sl Internet Stream Sockets}.

Les appels syst{\`e}mes marqu{\'e}s de <<\dag{}>> prennent en argument
une 
structure du type {\tt sockaddr\_un}.

Les appels syst{\`e}mes marqu{\'e}s de <<\ddag{}>> sp{\'e}cifient le
domaine
{\tt AF\_UNIX}.

Pour ce qui est des autres, le comportement est strictement identique
(m{\^e}me
arguments, m{\^e}me type de retour, etc.).

\begin{table}[hbtp]
\centering
\begin{tabular}{|l|l|l|}
	\hline
		Phase	&	Serveur					&	Client					\\
	\hline \hline
		1		&	{\tt socket()}\ddag{}	&	{\tt socket()}\ddag{}	\\
	\hline
		2		&	{\tt bind()}\dag{}		&	{\tt bind()}\dag{}\footnote{optionel}		\\
	\hline
		3		&	{\tt listen()}			&							\\
	\hline
		4		&							&	{\tt connect()}\ddag{}	\\
	\hline
		5		&	{\tt accept()}\dag{}		&					\\
	\hline
		6		&	{\tt send()}			&	{\tt recv()}			\\
				&	{\tt recv()}			&	{\tt send()}			\\
				&	{\tt write()}			&	{\tt read()}			\\
				&	{\tt read()}			&	{\tt write()}			\\
	\hline
		7		&					&	{\tt close()}			\\
				&	{\tt close()}			&					\\
	\hline
\end{tabular}
\caption{\label{unstream-syscall-recap}Tableau r{\'e}capitulatif des
appels syst{\`e}mes
	pour l'{\'e}tablissement d'un canal de communication entre deux
processus}
\end{table}

%%%%%%%%%%%%%%%%%%%%%%%%%%%%%%%%%%%%%%%%%%%%%%%%%%%%%%%%%%%%%%%%%%%%%
\section{Exemple}

Le but de ce programme est d'assurer un transfert de fichier entre deux
processus. 
\begin{itemize}
\item Le client envoie le nom du fichier et son contenu.
\item Le serveur re\c{c}oit ce nom de fichier et {\'e}crit le contenu
sur le disque.
\end{itemize}

Le programme serveur s'ex{\'e}cute comme un processus serveur {\Unix},
c'est-{\`a}-dire qu'il tourne en permanence et attend les requ{\^e}tes des clients.
Le seul moyen de l'arr{\^e}ter est de le tuer manuellement gr{\^a}ce {\`a} la commande
{\Unix} {\tt kill(1)}, ou bien, carr{\'e}ment, red{\'e}marrer la machine.

Lorsqu'une requ{\^e}te arrive, le serveur cr{\'e}e un sous processus pour traiter
la requ{\^e}te, et ainsi, continuer {\`a} prendre en compte les demandesd'autres clients.

%%%%%%%%%%%%%%%%%%%%%%%%%%%%%%%%%%%%%%%%%%%%%%%%%%%%%%%%%%%%%%%%%%%%%
\subsection{Partie <<~serveur~>>}
\begin{verbatim}
#include <sys/types.h>
#include <sys/socket.h>
#include <sys/un.h>
#include <stdio.h>
#include <fcntl.h>

#define BUFLEN 1000
#define QUEUE_DEPTH 5
#define NET_FILE "/tmp/asocket"

main ()
{
    int                 ls, nsock, client_len;
    int                 cpid;
    struct sockaddr_un  serv_addr, client_addr;
    
    /* Cr�ation du port de communication */
    if ((ls = socket(AF_UNIX, SOCK_STREAM, 0)) == -1 ) {
        perror ("serveur: erreur appel syst�me socket()");
        exit (-1);
    }
    
    /* Attributs du port de communication */
    serv_addr.sun_family = AF_UNIX;
    strcpy (serv_addr.sun_path, NET_FILE);
    if ( bind (ls, &serv_addr, sizeof(struct sockaddr_un)) == -1) {
        perror ("serveur: erreur appel syst�me bind()");
        exit (-1);
    }
    
    /* Param�trage de la file d'attente */
    if ( listen(ls, QUEUE_DEPTH) == -1) {
        perror ("serveur: erreur appel syst�me listen()");
        exit (-1);
    }
    
    /* On fait en sorte que le processus serveur tourne en permanence.
    ** On le rattache au processus syst�me Unix "init".
    **
    ** Le processus p�re cr�e un processus fils ind�pendant, charge le
    ** programme correspondant et se termine.
    */
    
    /* On fait en sorte que les processus fils soient ind�pendant du
    ** processus p�re.
    */

    setpgrp();

    if ((cpid=fork()) > 0) {
        /* pere */
        exit (0);
    }
    if (cpid == -1) {
        perror ("serveur: erreur appel syst�me fork()");
        exit (-1);
    }
    
    /* Boucle infinie pour le traitement des demandes de connexion */

    client_len = sizeof (struct sockaddr_un);
    
    while (1) {
        if ( (nsock = accept(ls, &client_addr, &client_len)) == -1 ) {
            perror ("serveur: erreur appel syst�me accept()");
            exit (-1);
        }
        
        /* Cr�ation d'un sous processus pour g�rer la connexion avec le
        ** client. Le processus pere continue � traiter les demandes
        ** de connexion qui arrivent.
        */
        
        if ( (cpid = fork()) == 0 ) {
            /* On charge le programme correspondant au traitement
            ** des requ�tes du client
            */
            request_clt (ls, nsock);
            exit (0);
        }
        
        if ( cpid == -1 ) {
            perror ("serveur: erreur fork() pour le traitement de requ�tes");
            exit (-1);
        }
        
        /* Fermeture du canal de communication */
        close (nsock);
    }
}

request_clt ( listen_socket, sk)
int listen_socket, sk;
{
    int     fd, len;
    char    dest[BUFLEN], buf[BUFLEN];
    
    /* Fermeture du canal correspondant � la socket en �coute, h�rit�e
    ** du processus p�re.
    */
    close (listen_socket);
    
    /* Reception du nom du fichier destination */
    if ( (len = read(sk, dest, BUFLEN)) == -1 ) {
        perror ("request_client: erreur de reception du nom de fichier");
        exit (-1);
    }
    
    /* Ouverture du fichier destination */
    if ( (fd = open (dest, O_WRONLY | O_CREAT, 0700)) == -1) {
        perror ("request_client: erreur d'ouverture sur la destination");
        exit (-1);
    }
    
    /* Lecture des donn�es du client et ecriture vers le fichier
    ** destination
    */
    while ((len = read (sk, buf, BUFLEN)) != 0 ) {
        if ( write (fd, buf, len) == -1 ) {
            perror ("request_client: erreur d'�criture sur la destination");
            exit (-1);
        }
    }
    
    /* Fermeture du fichier destination */
    
    close (fd);
    
    /* Fermeture du port de communication */
    
    close (sk);
}
\end{verbatim}

%%%%%%%%%%%%%%%%%%%%%%%%%%%%%%%%%%%%%%%%%%%%%%%%%%%%%%%%%%%%%%%%%%%%%
\subsection{Partie <<~client~>>}
\begin{verbatim}
#include <stdio.h>
#include <fcntl.h>
#include <sys/types.h>
#include <sys/socket.h>
#include <sys/un.h>

#define BUFLEN 1000
#define NET_FILE "/tmp/asocket"

int write_buf;

main (argc, argv)
int   argc;
char *argv[];
{
    int                 fd, sk, len, destlen;
    char                *source, *dest, buf[BUFLEN];
    struct sockaddr_un  peeraddr;
    
    if (argc != 3) {
        fprintf (stderr, "usage: %s source_file destination_file\n", argv[0]);
        exit (0);
    }
    
    source  = argv[1];
    dest    = argv[2];
    destlen = strlen (argv[2]) + 1;
    
    /* Creation du port de communication */
    if ( (sk = socket(AF_UNIX, SOCK_STREAM, 0)) == -1) {
        perror ("client: erreur appel syst{\`e}me socket()");
        exit (-1);
    }
    
    /* Envoi d'une demande de connexion au serveur */
    strcpy (peeraddr.sun_path, NET_FILE);
    peeraddr.sun_family = AF_UNIX;
    
    if ( connect(sk, &peeraddr, sizeof(struct sockaddr_un)) == -1) {
        perror ("client: erreur appel syst�me connect()");
        exit (-1);
    }
    
    /* Envoi du nom du fichier destination */
    
    if ( write(sk, dest, destlen) == -1 ) {
        perror ("client: erreur d'�mission du fichier destination");
        exit (-1);
    }
    
    if ( (fd = open (source, O_RDONLY)) == -1 ) {
        perror ("client: erreur d'ouverture de la source");
        exit (-1);
    }
    
    while ( (len = read(fd, buf, BUFLEN)) != 0) {
        if ( write (sk, buf, len) == -1 ) {
            perror ("client: erreur d'�mission des donn�es");

            exit (-1);
        }
    }
    
    /* Fermeture du fichier source */
    close (fd);
    
    /* Fermeture du port de communication */
    close (sk);
}
\end{verbatim}